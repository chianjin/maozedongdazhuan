\documentclass[../../dazhuan.tex]{subfiles}

\begin{document}
\clearpage
%\pdfbookmark{第三版絮语}{v3say}
\chapter*{第三版絮语}
\addcontentsline{toc}{part}{第三版絮语}

一、初衷。笔者还在开封师范学院(今河南大学)读书时,曾与一老友商量,将小说《海岛女民兵》改编成电影脚本,未成;后又商量将小说《桐柏英雄》改编为戏剧剧本,又不成。上世纪七十年代末,妻下岗,一家4口人只有我36元的月薪,无法养家,只得在工作之余辅助妻做一些小生意,到八十年代初,生活始得自足,便着手从《新唐书》、《旧唐书》、《新五代史》、《旧五代史》、《资治通鉴》等史书中整理有关黄巢起义军及朱温、李克用、李存勖的资料,历数年写成《晋阳英烈》上下卷约五十万字的初稿,旨在用历史探原小说的形式,说明黄巢起义失败的主要原因,暴露农民起义军叛徒朱温的狡黠、奸诈、凶残和丑陋,表现少数民族杰出人物李氏父子对大唐王朝的感恩和赤诚。到了八十年代末、九十年代初,有关毛泽东的书籍开始上市,我便着手研究,发现凡大部头的书籍,无一不说毛泽东在社会主义革命时期怎么怎么错了。正如我在第一版序言里所说的原因,便将处女作《晋阳英烈》搁置一旁,开始构思《毛泽东大传》了。每天处理完公事,就坐下来整理资料,晚上一直干到后半夜。退休以后,全身心地投入写作,一天要坐上十几个小时。有朋友劝我:“何苦呢?”我戏言:“早有写书意,半生不得闲。一发不可收,拼着老命玩。”

二、构思。扉页毛泽东语录一,高度概括了毛泽东的一生。毛泽东后来把这几句话发展成为“阶级斗争、生产斗争、科学实验”三大革命实践。扉页毛泽东语录三,体现了毛泽东的伟大谦虚和高洁的品格。
本传分卷是按照毛泽东一生中的各个阶段来划分的。他生命中的每一个阶段,都有一个中心任务,或学习或奋斗目标,而中心任务在各个阶段中无不一以贯之,而各个阶段的中心任务又截然不同,故本传将其一生分为10个阶段,一阶段为1卷,全书共10卷。每一卷的卷目都是4个字,均取自毛泽东的诗或词中。每一章正文前面采用该时期毛泽东的一句或一段名言作为章目,这是受了《斯巴达克斯传》一书的启发。这样的形式不是笔者有意标新立异,而是因为每一章的内容都很多,不可能用传统的对子句式来高度概括。本传如此安排,使每卷卷目和每章章目与本传的主旨紧紧相扣,全书浑然一体。更重要的是,毛泽东在各个时期的部分著名论断赫然在目,更能突出本传的主旨。  

三、全书用编年体,全面而详细地记录了毛泽东的生活和工作,是学习毛泽东著作和研究毛泽东思想的最好的辅助参考资料。《毛泽东大传》和毛泽东著作相辅相成,再现一个真实的、鲜活的毛泽东形象。当然,写毛泽东的历史采用时空倒置、时空穿插的叙述方法,也不是不可以的,但这种方法往往成为一些别有用心的人篡改历史、颠倒历史的惯用伎俩。这样的写法,看似花哨,显得高深,别具一格,却使人看了眼花缭乱,如坠雾中。更重要的是,那些别有用心的人,可以假借此种手法颠倒黑白,甚至移花接木,篡改历史,以彰显自己的胡说八道。有一位比较有名气的作家在一本将军传记中,就曾经采用了这种时空倒置、时空穿插的方法,发出了一些不好的议论,甚至把毛泽东描绘成为一个阴森森的人,借以影射毛泽东是一个惯用权术的政治家。

四、在毛泽东大传的编撰过程中,笔者时时有感而发,不吐不快,那便是传文中的“正是”、“这正是”和一些章尾的“东方翁曰”一类文字。前者是受了元明清小说的影响,后者是受了太史公司马迁的影响。《史记》的优点之一就是史论分离,“太史公曰”既抒发了史学家的情感,又不影响正史。而元明清小说中的诗、词、小令,则是既抒发了作者的情感,又不影响故事的情节,同时还增加了一点可读性。笔者自幼读书,深有感受,东施效颦,并非故意卖弄也。

五、参与第3版稿纠错的有刘玉民(郑州 教授)、王孝民、徐君兴、郝长臣(大庆)、曹子文(天津)、邢思明诸同志及一些不愿意提供大名和遗漏了大名的良师益友,在此一并致谢!还必须提及的是河南省孟津县教师进修学校退休的徐君兴同志,他在2015年3月4日和3月20日给我写了两封热情洋溢的信,像老同志、老朋友一样,谈到了关于《毛泽东大传》的修改意见,其中纠正了再版《毛泽东大传》中数千个错别字,使我获益匪浅。没想到仅仅数月后,我在与李德桥同志的通信中,惊悉徐君已于10月7日病故,甚为痛悼。自此以后,我就把徐君兴同志的两封信摆在我的电脑旁,时时看到,时时激励着我修改《毛泽东大传》。我相信,徐君的在天之灵见到毛主席之时,老人家一定会高兴地说:“徐君兴是个好同志!”

六、今年是毛泽东同志逝世40周年,是我在1966年11月3日接受他老人家检阅的50周年。那是毛主席第6次接见红卫兵。我这个从父辈往上查4代人都不识一个字的农家子,作为毛主席的客人在煤炭科学研究院吃住12天,平生第一次享受到餐餐8大盘、主食随便吃的特殊待遇,每天晚上还能在大厅里看电视,白天乘着汽车到北大、清华等各大高校参观“文革”大字报。几十年以来,我一直把11月3日作为我的节日。在毛泽东逝世40周年纪念日和11月3日这个刻骨铭心的“一饭之恩”纪念日即将到来之际,特将《毛泽东大传》第3版(471万字,全6册)付梓发行,敬献给广大的读者朋友们,并恳请有关“毛学”专家、学者和同志们继续帮助修改本传,使之日臻完善,成为没有校园的毛泽东思想自修大学,吾愿足矣!

\kaishu
立言五百万,坦荡无私偏。歌颂真善美,鞭笞假恶奸。

一代风云录,明镜照人寰。掩卷聊自慰,痴心敢对天。

\hfil ——六十七周岁感怀 赠读者
\normalfont

\hspace{18em} 东方直心

\hspace{18em} 2016年1月
    
\end{document}