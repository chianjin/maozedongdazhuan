\documentclass[../../dazhuan.tex]{subfiles}

\begin{document}
\clearpage
%\pdfbookmark{第四版说明}{v4des}
\chapter*{第四版说明}
\addcontentsline{toc}{part}{第四版说明}

{\centering \zihao{4}\kaishu 一 \par}                            
                        
毛泽东大传第三版在2016年1月付梓之后,笔者便开始了第四版的修改工作,重点放在了修辞、改错字,并逐一核对资料、纠正讹误,收集、增补新的原始资料等方面。所增补内容,主要是毛泽东的电报原文及其重要文稿等,为“毛学”研究者下载资料提供了方便。

第四版文稿已近700万字,基本上消灭了错别字,在资料方面纠正了讹误,在修辞方面也有了很大的变化,还在每一章的末尾补齐了笔者的议论部分(即东方翁曰)。可以毫不夸张地说,这个新版本已经成为“到目前为止在国内外可以见到的所有毛泽东传记中唯一一部最系统而不凌乱、最全面而不单一、最客观而无片面性、最真实而不胡说八道、最具可读性而不枯燥的信史”。只要您认真地读完了本传,一个活生生的毛泽东的光辉形象,就一定会显现在您的面前,将使您铭刻在心,终生难忘!

还需要说明的是,笔者始终不赞成神化毛泽东,那些把毛泽东说成是“神”,把毛泽东宗教化的做法,是没有道理的,是不可取的。毛泽东是人,不是神而胜于神。他是伟人,他是伟人中的伟人!神能救中国吗?神是爱人民的吗?不!请你翻开历史看一看吧,请你到寺院里考察一下吧!神是不能救中国的,神是喜欢有钱人的!君不见,毛泽东比神厉害吗?他不但挽救中国于危难之世,还在列强的虎视眈眈环伺之下,建设起了一个强大的社会主义中国!这是任何一位神仙都做不到的!看完了毛泽东大传,你会感觉到毛泽东更像高堂慈母,更像一位严父。慈母之心,严父之爱,在他身上常常体现出来。每临大事,他都不厌其烦地对部下反复讲,还要到处讲,反复交代,反复叮咛,这不正是“母亲的唠叨”吗?部下犯了错误,除了严厉训斥,他还会罚站,尔后免不了来一番劝告和慰勉,这不正是活脱脱的严父吗?\emph{双亲健在之时,身为儿女,尚不觉得父母有什么特别可敬可爱之处,父母一旦故去,他们的养育之恩、训教之功,便时时萦绕心头,挥之不去。}同样,毛公在世之日,他的部下,他的人民,也有不少人觉得政治运动太多了,嫌这老头多事。\emph{不是说“六亿神州尽舜尧”嘛,哪里有什么阶级敌人呢?怎么还会受二茬罪、吃二遍苦呢?现如今人们在毛公像前顶礼膜拜,痛哭流涕,方知老人家的“慈母之心、严父之爱”是多么的可贵!}除了他,哪路神仙能救你脱离苦海呀,哪个菩萨能普度众生呢?!

有不少网友问我,毛泽东到底有没有缺点?我的回答是两句话:一是金无足赤、人无完人。二是全世界的反动派都在骂他老人家,你再说什么缺点,岂不是和反动派同流合污了吗?\emph{还是好好地读书吧,读毛泽东,读毛泽东思想,脱离愚昧,永远做一个不糊涂的明白人,是为至要!}

\sign{东方直心\\2021年1月}
\par\mbox{}

{\centering \zihao{4}\kaishu 二 \par}   
 
(\emph{一})参与第四版纠错的有吴炳奎(丹东)、孔繁铭(福建厦门)、王裕隆(上海闵行区)、任玉龙(河南安阳)、王建叶(河南洛阳)、李德桥(河南孟津 原公社书记)、张哲、梁治洲(上海闵行区)、王洪海(山西太原)、聚沙斋主人(浙江湖州)、欧建国(湖南湘潭)、荆庆喜(河南新郑)、张晓平(黑龙江牡丹江)、张西月(河北石家庄 教授级高工)、井冈山观心(北京海淀)、李晓军(陕西西安)、王润东(河北石家庄)、李甫(浙江杭州)、王欢(广东深圳)、陈一豪(河南灵宝 在校大学生)、易楚(北京 图书编辑)、丁宏学(北京 律师)、田创(陕西咸阳)、侯丽军(河北邢台)、王宏恩(江苏泰兴)、吕天宇(北京)、潘继开(浙江宁波)、魏明明(江苏徐州)、路炳银(河南郑州)、余迅(安徽潜山)、熊林平(上海)、苏向新(泉州)、牛旭林(河南唐河)等诸位朋友,笔者在此表示深切感谢!\emph{在上述良师益友中,更有聚沙斋主人、张西月、井冈山观心、王润东、李甫、陈一豪6位先生和朋友,不辞劳苦,经年累月地审阅电子稿,助我纠错。他们的无私奉献,为促使第四版文稿日臻完善,的确与有功焉。特别是陈一豪同志,利用电脑软件对本传中讹误之处扫荡式地搜索,更是功不可没。}

(\emph{二})笔者为准备、构思、写作、修改本传,历经三十年,无论寒来暑往,也无论逢年过节,几乎是天天“笔耕不辍”;在退休前后相当一个时期,更是废寝忘食,每晚不过午夜二时不入睡;现在已经说不清楚究竟修改、补充了多少遍,这才圆满地完成了700万字的《毛泽东大传》终极版。让《毛泽东大传》成为政治家的范本,军事家的教科书,历史学家不可或缺的重要资料,年青一代的成长指南,是我编纂本传的初衷,也是我一生中最重要的心愿。其间,鄙人虽然落下了难以治愈的顽疾,亦不足为憾也。掩卷细思,历史悬疑尽释,一所没有校园的“毛泽东思想自修大学”已经达到了出类拔萃之化境!更有电子版广泛传播,“毛学”无处不在!甚慰,甚慰!今生无憾,俯仰无愧,吾愿足矣!

\mbox{}\par
\sign{东方直心\\2023年2月}    
\end{document}