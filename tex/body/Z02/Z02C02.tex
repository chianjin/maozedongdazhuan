\documentclass[../../dazhuan.tex]{subfiles}
% 寸草春晖
\begin{document}
	\chapter*{沟北人}
	\addcontentsline{tod}{chapter}{沟北人}
	\pdfbookmark{沟北人}{Z02C02}
	
我祖上原是山西洪洞人氏,始祖尊讳单名一个“伟”字,大约在十八世纪中后期带领他的五个儿子(广德、灼德、驰德、崇德、另有一人轶名)迁来河南宛城东45里的地方落脚。

二世祖以“德”字排辈,后人称为“老五门”。遗存有“灼德(字华道)碑”(同治十三年即1875年小阳月立)、“崇德(字徒义,行五)碑”(道光二十四年即1845年重阳月上旬立)。

三世祖以“南”字排辈,遗存有“图南(字志远)碑”(清故太学生,咸丰六年即1857年清明节立)。

这三座石碑的碑文记载了祖上当年人丁之盛。

始祖和二世祖父子六人,在村子西边驻扎下来,与村子东边几家同姓不同宗的原住民比邻而居,倒也相安无事。他们父子努力开拓,主要向着村北、村西、村西南近距离没人烟的地方发展,很快便拥有了十几顷肥沃的黑土地,不但建起了豪宅(我童年时还有两座比肩而建的一进三的古建筑),而且还在豪宅的东北角修筑了一座炮楼,在村外东北角建起了一座庄严的青砖灰瓦关帝庙,使整个村庄与东边和北边近距离(约3里左右)的两个大村落形成了鼎立之势。

始祖颇有经济头脑,不但广开良田,而且还扩展到宛城经商,占了半条街。他老人家亲自坐镇,开起了粮行等商店。每逢过年过节,还有他的寿诞之日,儿孙们便用他那驾特制专用的豪华大马车,将他接回村子里热闹一番。始祖在世之时,生产资料和财富是公有制,一家人祖孙三代过的是“原始共产主义”生活,其乐融融。

始祖去世后,公有制瞬间坍塌。“老五门”分了家,各自另立门户,私有制取代了公有制。在这个大家族中,逐渐出现了贫富分化。

据“灼德碑”记载:灼德(字华道)公在分家后,“人皆尚华,公独率真”,“遭时不偶(汉语成语),坎坷终身。设教闾里,诲人谆谆。至老不倦,年近大椿”。他“所生四子,文质彬彬,咸通医书,济世活人。”由此可见,贫富分化始于二世祖那一代人。他这一支已经从“望族”中分离出来,转而成为“中产家庭”了。

另有“崇德碑”、“图南碑”两座,从另一个侧面证实了本家族的阶级分化状况。老五崇德唯一的儿子在“崇德碑”上被称为“监生图南”(监生即国子监生员)。“图南碑”上称其为“清故太学生(在古代最高学府国子监就读的学生)”。图南娶了两房妻子,长房李氏,育有一子,单名一个“兰”字(庠生,即秀才。碑文称其为降服男,意思是丧服降低一等。比如为父母应服三年之丧,其已出继者,则为本生父母降三年之服为一年之服。兰有二子,辉堂、辉第,均为庠生,即秀才)。图南第二房妻子刘氏,育有六男,金字旁排辈。碑文称之为奉祀男(供奉祭祀父母的儿子)。

图南是整个家族中学历最高者,不幸遭遇第二次鸦片战争(1856年10月-1860年10月)爆发,辍学回到故里,于次年(1857年)病逝。图南的学历、婚姻和后人,说明他这一支人的身份、财富和地位远远在灼德那一支人之上。

我嫡亲的祖宗(三世祖)叫双南(据传是二世祖广德之子),于某年夏天,在西河大桥下乘凉时,与人赌博,以家产为赌资,在西瓜皮壳里斗骰子(又称掷骰子、投骰子),一下子把家产输了个精光,成为本家族内第一个贫困户,比灼德一支“中产家庭”更惨了。这是始祖伟公后人中出现贫富分化的又一个典型事例。

祖上因赌败家后,只得搬出大院,在村子西北角搭了几间茅屋,安顿下来。这几间茅屋孤零零地与村子隔了一条小水沟,所以人老几代都被村里人贬称为“沟北的”,或者直接叫“沟北人”。

双南公有三个儿子(即我的四世祖辈),只有一人诞育了我们这一支人(其余二人一人南迁到了汉塚以南的张庄,一人迁移到了东北方向),父亲只记得他叫“老张金”(不知行几)。前文已经说过,“灼德碑”记载,这一辈人姓名为两个字,名字是“金子偏旁”(灼德七个孙子中有二人“名”不带金字旁,说明此时已经乱起名字了)。大概是人穷也不讲究了,我“四世祖”的名字,干脆不要“金子偏旁”了,就叫一个“金”字。

“老张金”也有三个儿子(即我的五世祖,也就是曾祖辈)。这辈人姓名三个字。据“灼德碑”记载,他们姓名末尾都是一个“中”字(灼德九个曾孙名字都是如此,如“信中”、“用中”、“望中”等)。父亲早就忘了我曾祖、二曾祖、三曾祖的名字,只记得三曾祖的绰号叫“三瞎子”(大名如中)。

曾祖和二位曾叔祖年幼时,每逢过春节,他们的父亲“老张金”就说带他们到地里去拾柴禾。到了地里,老人说,你们玩吧。不是让你们来拾柴,你们没有新衣服穿,和人家的孩子在一起,怕人家笑话。

常言说,人穷被犬欺。现在还可以看到这种情景,狗见到穿着破烂的人,就以为是讨饭的,非追上去狂吠一番不可。那时候,全村只有一口水井,三个曾祖去挑水,被人断了路。年轻气盛的曾祖父,带着两个弟弟,各执一根白蜡杆子,骂骂咧咧地去拼命,吓坏了歹人,这才又争得了挑水的行路权。

正是由于贫穷,自我父辈往上查,四代人都不曾上过学,斗大的字不识一个。祖坟上无钱立碑,辈辈人没文化,几代人的历史都是口口相传,久而久之,连祖上的名字都忘了。

列宁说:“小生产是经常地、每日每时地、自发地和大批地生产着资本主义和资产阶级的。”

熟读列宁的毛主席也深知,小农经济私有制是人与人之间产生贫富分化的根源。因此,他在中国社会主义公有制条件下,多次强调反对包产到户,坚决反对分田单干。

正是在那个小农经济时代,我祖上族人之间亲情日渐淡薄,代之以尔虞我诈,坑蒙拐骗,偷抢奸杀,无恶不生。贫富差距越来越明显,阶级分化越来越严重,族人中形成了不同的阶级。一些贫困子弟,无以糊口,失去生计,便纷纷沦为土匪、恶霸,抢劫、绑票的歹人。其中著名的有抱定“兔子不吃窝边草”宗旨,跑到西北山区安营扎寨,震慑于百里之外的大杆子头荣庆。有双手打枪、百步穿杨,在国民党监狱中几进几出的黑道传奇人物“赤巴脚”。而那些小盗毛贼,偶尔也杀人越货,在方圆地界上颇有恶名。人言“嫁女不嫁某堂汉”,便是明证。

在我的童年,族人已经繁衍至七八百人,村中仅有一家大地主。在解放前夕,他们弟兄之间产生矛盾,打官司数年,弟弟死在监狱里,这一支人后来成了贫农。老大夫妇都是北京大学毕业生,这一支在解放后,作为大地主,受到了惩罚。除了他们之外,还有两家小地主,少有富农,其他绝大多数族人不是富裕中农,就是贫农、下中农。

这个大家族的历史,实际上就是一部阶级斗争史。由此可见,私有制乃万恶之源。
	
\end{document}