\documentclass[../../dazhuan.tex]{subfiles}
% 寸草春晖
\begin{document}
	\chapter*{曾祖母的功德}
	\addcontentsline{tod}{chapter}{曾祖母的功德}
	\pdfbookmark{曾祖母的功德}{Z02C03}
我曾祖父膝下有二男二女,他们是大祖父光运、祖父光山(即六世祖,我祖父行二)、二姑奶、三姑奶。二曾祖膝下有一男一女,他们是三堂叔祖光志、大姑奶。三曾祖膝下有三男一女,他们是四堂叔祖光海、五堂叔祖光兴、六堂叔祖光甫,四姑奶。

早在十九世纪末,我们这一支人第一次分家。二曾祖膝下一男一女长大成人,另立门户。

1915年(民国四年),我祖母和大祖母相继去世。那一年,大伯13岁、二伯9岁,二姑尚幼,我父亲才一岁多。在那个积贫积弱的年代,穷人是福无双至祸不单行。不久,嫁入邻村田氏的三姑奶也英年早逝,留下了一男一女比我父亲还小的两个幼儿,也送至曾祖母膝下抚养。一窝6个没娘儿,嗷嗷待哺,曾祖母日夜操劳,艰辛备尝。

1921年(民国十年九月初九日),本支人第二次分家。三曾祖膝下三子长大成人,另立门户。

(1933年,本支人第三次分家。大伯父、大伯母及大姐、大哥成恩另立门户。1935年,本支人第四次分家。二伯母被迫放弃继承过继父母的祖业、带着儿子改嫁国军一个岳姓班长,去了数十里开外的唐河县白秋镇。)

我曾祖母是一位持家好手。二位祖父一年到头辛勤劳作,收获总是难以维持一家人的生计,全靠曾祖母千方百计调剂,省吃俭用,糠菜参半,全家人总算饿不了肚子。

1928年(民国十七年),全家人倚为主心骨的曾祖母一病不起,于农历七月初一日溘然长逝,享年71岁。

父亲幼年常常依偎在我曾祖母怀中,显得特别可怜,不但处处受到曾祖母呵护,就是当家的大祖父对父亲也疼爱有加。父亲说起曾祖母和大祖父,至今还是一往情深。父亲说,我们这一支能够延续下来,全靠了曾祖母。所以,父亲每年上坟祭祖,总要给曾祖母和大祖父多烧上几张纸钱。父亲对曾祖母的养育之恩,可谓没齿不忘\footnote{上文写于1990年6月27日。}。

\begin{poem}*[8em]{祭曾祖母文}
曾祖王母\footnote{即曾祖母,旧称曾祖王母。},汉塚邢氏。

幼适吾门,育四子女。

家道贫寒,相夫教子,

勤勉劬劳,困顿异常。

娶媳嫁女,中兴有望。

天不佑我,二媳一女,

姑嫂不寿,英年早亡。

六小儿女,王母哺养。

将雏抚幼,艰辛备尝。

哺之育之,六孙渐长。

繁衍生息,渐成大户,

三姓\footnote{弓长氏、田氏、二姑适秦氏。}感恩,子嗣绵长。

呜呼!斯世之惨淡兮,问世间有几?

曾祖母功德之高兮,亘古可有其匹?

母爱之伟大兮,万古流芳。

五世之戴德兮,高山景行。\footnote{初稿于1994年1月9日。}
\end{poem}

\end{document}