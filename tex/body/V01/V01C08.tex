\documentclass[../../dazhuan.tex]{subfiles}
% 第一卷
\begin{document}
\chapter*{第八章}
\pdfbookmark{第八章}{V01C08}
\begin{pref}
	清水池中蛙句句,为公乎,为私乎?
\end{pref}

话说在1917年夏,萧子升请毛泽东为其学习笔记《一切入一》作序。毛泽东欣然命笔,写出了自己治学的经验,他写道:

“吾生也有涯,而智也无涯。今夫百丈之台,其始则一石耳,由是而二石焉,由是而三石、四石以至于万石焉。学问亦然。今日记一事,明日悟一理,积久而成学。台积而高,学积而博。”

这一时期,他还写有这样一副有关学习心得的对联: 

\begin{couplet}
贵有恒,何必三更眠五更起;

最无益,莫过一日曝十日寒。
\end{couplet}

毛泽东不仅刻苦攻读“死书”,而且还善于读“活”的书本。他常说要读有字之书,还要读“无字之书。”因为“闭门求学,其学无用。”

有一天,毛泽东从《民报》上读到一则消息,说是有两名中国学生自费旅行全国,一直走到了西藏边境的打箭炉。这使他受到很大的启发和鼓舞。他很想效法这两名学生,可自己口袋里却没有钱,想来想去,只能采用“游学”的方式,先在湖南游历一番。

所谓游学,是在旧社会中一些有志读书的穷书生用寻师访友获取资助的方式,以达到求学的目的;也有一些没有出路的读书人,他们既贫困又不愿意从事体力劳动,便利用“游学”的方式写字作对联送人,靠这种变相的行乞过活。

1917年7月中旬,毛泽东利用暑假,邀同在楚怡小学教书的老同学萧子升和准备回安化老家度假的同学萧蔚然,在这稻谷吐金山果溢彩的季节里开始了第一次游学。

毛泽东身上分文未带,只穿着一件白色旧上衣,带着一把旧雨伞和一个布包,布包里装一套换洗衣服、毛巾、笔记本、毛笔和墨盒,与萧子升、萧蔚然一起从楚怡小学出发,经长沙小西门渡湘江,经白若铺,向宁乡县城走去。一路徒步走来,走着走着,他们就感觉饥饿了,而且是越来越难以忍受了,便打听附近有没有什么文化名人。路旁的人告诉他们说,附近住着一位姓刘的老乡绅,是前清的翰林。萧子升高兴得嚷了起来:

“润之,刘先生今天就是我们要去见的主人了!我们首先向他进攻。我想最好的办法是写一首诗给他,用象征的语言表示我们拜访他的目的。”

毛泽东十分赞同,他说:

“好主意。让我们想想。这首诗开首第一句可以是:翻山渡水之名郡,”

萧子升说:

“很好!第2句:竹杖草履谒学尊。接下去可写:途见白云如晶海,”

毛泽东环顾了一下他们三人的形象,笑着说:

“末句可是:沾衣晨露浸饿身。”

诗联成了,他们又细细读了几遍,感到相当满意。毛泽东高兴地说:

“刘翰林应该佩服我们的勇气。我们马上就能看到他到底是怎样的一位学者了。”

三人一路打听,来到刘家宅院,70多岁的刘姓老翰林接待了他们。毛泽东说明来意,萧子升借着书案将联诗誊写好,呈给老先生。刘老先生看他们的诗写得好,书法也挺不错,不但款待了他们,还赠了他们40枚铜元。

毛泽东和萧子升、萧蔚然辞别了刘老翰林,又去欣赏了城郊玉潭河边的风景。这玉潭河比较宽,河上有一座精巧的小桥,桥附近聚集着许多小船;站在河边远远望去,可以看见一座小山岗,那就是狮固山,山坡上长满了大大小小的松树,生机勃勃,景致如画。

毛泽东、萧子升坐在河岸边,观赏着玉潭河和周围美丽的景色,诗兴顿起,又联诗一首:

云封狮固楼,桥锁玉潭舟……

很遗憾,后来萧子升在回忆此事的时候,只记起了以上这两句,因之此诗便成了残篇。

且说毛泽东一行三人到了宁乡县城,在同班同学王熙家里住了两天,走访了劝学所和玉潭高小,还到香山寺游玩了一番。毛泽东在离开王熙家时,留赠对联一副,他写的是:

爱君东阁能延客;别后西湖赋予谁。

毛泽东三人来到宋家潭,找农民宋冬生了解了生产和生活情况;又走访了一位老先生,同他探讨了一些有关经书方面的问题。他们上了回龙山,给白山寺的和尚送了一副对联;在黄材镇了解了农村小市镇的贸易情况,还为一些店铺书写了招牌。尔后他们经横山湾步行100多公里,来到杓子冲何叔衡的家里。

前面已经说过,何叔衡曾是毛泽东在湖南公立四师和一师的同学,他于1914年7月提前毕业后,受聘于长沙楚怡学校,任主任教师,和萧子升是同事。

毛泽东一直与何叔衡保持着密切的联系;他们的到来,受到了何家人的热情款待。毛泽东观看了何家的猪栏、牛栏、菜园和稻田,了解了何家的经济状况和家庭历史。经何叔衡介绍,他还走访了何叔衡的堂兄弟和附近的农民。在一户贫苦农民家的桐油灯下,主人告诉毛泽东说,他是佃农,交的是“三七租”,一年累到头,生活苦得很,遇到灾荒,日子就更难过了。

在杓子冲附近的宋家潭,毛泽东又访问了一位老农和一位青年农民,听了这些贫苦农民的诉说,对中国农民的状况有了更深切的了解。

毛泽东自游学以来,每天早起都要到野外做“六段操”,然后是看书,追记笔记,在何家期间也不例外。何叔衡的父亲见他如此,高兴地说:

“毛先生能文能武,玉衡跟着他会有出头之日的。”

毛泽东三人离开何家时,何家人要送他们一些路费,毛泽东坚辞不受。他们向着沩山方向信步而行,经沙田到巷子口,一路上海阔天空地聊着,不知不觉来到一座不知名的小山上。山坡上有一棵古老的松树,长势茂盛,他们卸下包袱,坐在岩石上,背倚古松,讨论着中国农民的生活状况,说着说着就困了,竟然都睡着了。一觉醒来,大家都感觉肚子饿了,便向山脚附近的一座房子走去。房主人拒绝给“叫花子”饭吃。他们无奈,只得继续向前走,走了约摸一里路,来到一户人家,一对和善的老夫妇给了他们足够的饭菜。

这家主人姓王,问道:

“你们看上去并非乞丐,天气这么热,你们出来干什么?”

毛泽东回答说:

“我们的家境不好,都很穷。但我们想旅行,看看农村的情况,没有路费,只好游学。”

老人说:

“游学并不坏,游学的人往往是正直的,诚实的。只有那些当官的人最不正直,多数人也都不廉洁。我曾经在衙门里当过门房,亲眼看到,谁想打赢官司,谁就得送钱送礼,谁送的多,谁就可以打赢官司。这就叫‘衙门八字开,有理无钱莫进来’。如果输官司的一方告到省城,那就更没有打赢的希望了,因为在省城里打官司,比在县城里花费更大。在县城都没有足够的钱去行贿,哪有足够的钱去省府行贿更大的官哩。官官相护,谁人不晓。”

老人的话引起了3个年轻人的联想,他们深感人世间之不公平。毛泽东愤然说:

“这是什么世道!”

告别老人后,毛泽东心情依然沉重,一路上与萧子升,萧蔚然讨论着社会上那些贪官污吏的龌龊事。3人来到宁乡县沩山脚下,暮色已经降临了。

沩山是佛教史上的名山,山上的密印寺是自唐代起就很有名气的佛教寺院。寺院里面住着一百五六十个和尚,毛泽东很想了解一下僧侣的生活状况。

山门前的两个和尚打量着毛泽东3人的装扮,以为他们是远道而来的香客,便热情接待他们,其中一个和尚还陪着他们进了寺院。毛泽东见和尚如此热情,为免除误会就告诉陪同的和尚说:

“我们本是为乞讨而来的。”

那和尚听他如此说,也就随口道:

“拜佛和乞讨本就是一回事。”

毛泽东3人随着和尚穿过二门,来到后面的禅院,看见院子里约有百余名僧人在踱步,便一边走一边有礼貌地与路边的僧人打着招呼。和尚把3人领进一间禅房,叫他们放下包袱去沐浴。毛泽东等高高兴兴地洗了一个澡,刚穿好衣服,和尚又来了,说是让他们去进香。毛泽东解释说:

“我们并非为进香而来,只是为了见见方丈。”

和尚重新打量了他们一番,说道:

“方丈一般是不会客的。在方丈讲经说法时,你们也许能够看见他。”

毛泽东说:

“我们不但想见见他,而且还想和他谈谈,就在今晚。”

毛泽东说着递给和尚一张便条,让和尚交给方丈。大约10分钟后,和尚回来说:

“方丈现在就请你们过去。”

毛泽东3人随着和尚进了方丈的禅房,只见那方丈年约五旬上下,面目还算和善。这禅房的四壁都摆满了书籍,除了佛教经典之外,还摆着老子和庄子的著作;禅房中央的桌子上摆着一高一矮两个花瓶,高的装着鲜花,矮些的装着兰草,此外别无它物。毛泽东3人与方丈谈论了一些中国古典经典著作,方丈显得很高兴,便邀他们共进晚餐,无非是一些素食之类。

聚集在大殿里的许多僧人,见毛泽东3人从方丈的禅房里出来,就把他们当成了方丈的贵客,而且看样子还是很有文化的人。不少僧人纷纷拿出扇子或经卷,请他们题字留念。毛泽东和萧子升已经练就了一手好字,也不客气,提起笔来着实忙活了好一阵子。他们又参观了佛殿,饶有兴趣地观看了历代僧人的遗迹和遗物。

次日一早,毛泽东3人正要离去,和尚来说:

“方丈挽留各位施主再小住几日,下午还要再见见你们。”

毛泽东等人闻言,正合心意,便高兴地留了下来。他们在和尚的引领下,先后观看了寺里的菜园子、大厨房、斋堂等各个地方。

这天下午,毛泽东3人再次来到方丈的禅房,与方丈谈了一些佛教的善事,又谈到了他们熟悉的孔子和老子。毛泽东很想了解一下寺里的情况,就问方丈,这里有多少僧人?方丈说,大约有百余名是属于本寺的,其他的都是来自远方的游僧。平时寺里住有三四百人,前些年一度住过800余人,这是建寺以来最多的了。不过这些游僧一般住上几天,就又上路了。毛泽东又问:

“数千里之外的和尚,为什么会跑到这里来呢?他们来这儿干什么呢?”

方丈说:

“他们是来听经和受戒的。本寺方丈素以说法著名,全国僧人大多都知道本寺。这里寺产甚丰,招待他们停留数日,是不成问题的。你们也知道,和尚是出家人,所有的寺院都是他们的家,云游四方,在寺内谈经论道,彼此都能得到启示。”

毛泽东闻听此言,想到自己征友的目的不也正是要和朋友们“谈经论道”,彼此启发吗?于是便频频点头。只听萧子升问方丈道:

“全国有多少和尚?”

“这没有确切的数字,”方丈解释说:“除蒙古和西藏之外,全国至少有几万和尚。蒙古和西藏的僧人比例极高,若加上他们,恐怕就有几十万,甚至更多。”

萧子升又问:

“全国有多少像沩山这样的讲经中心?”

方丈说:

“至少也有百余处。如果算上规模较小的地方,那大约有千余处左右。”

毛泽东问:

“有什么佛教方面的书籍出版吗?”

方丈说:

“有的,而且还很多,尤其是在上海、南京、杭州这样的讲经中心。”

萧子升说:

“我们打算探访一些大寺院,您是否可以给我们写一些介绍信?”

方丈说:

“这不必要。你们不需要任何介绍信。因为你们到任何一处,都会受到像在这里同样的欢迎。”

毛泽东3人向方丈道了谢,说明日就要动身了。方丈说,施主们既然要走,老衲也不便强留。只希望在你们离去前,再见上一面。毛泽东说:

“我们喜欢一早动身,就不再叨扰了。”

次日清晨,毛泽东3人离了密印寺,走出了宁乡县境,到达安化县的司徒铺。这里是萧蔚然的家乡。萧蔚然辞别两位好友,独自回家去了。毛泽东和萧子升二人继续前行,来到伏口罗驭雄同学的家里。他们在罗家吃过中饭,就上横坡仑去了久泽坪,给当地秀才吴幼安送了一副对联,讨得一些资费。二人又经清塘铺、太平段,前往安化山区的县城梅城。是日晚,他们就露宿在河堤上。毛泽东风趣地说:

“沙地当床,石头当枕,蓝天为帐,月光为灯。”

他还指着身边一棵大树说:

“还有衣柜。”

说着就把衣服和随身带的东西挂在树枝上。临睡前萧子升还要到河下洗洗脚,毛泽东笑着说:

“你还要保持那绅士习惯。你是一个要饭的绅士哩!今晚试试不洗脚看能否睡得好。”

这萧子升一路上都放不下他那书生架子,每每向人问路之时,总是要先整整衣服,干咳两声然后才开腔。而且他每到一地,还只愿去和那些大户人家打交道。

第二天,毛泽东和萧子升在一家小饭店吃饭之时,听老板娘诉说了她家的苦难遭遇。他俩打听到附近有一座刘邦庙,又对刘邦的评价问题进行了讨论。饭罢,二人边走边谈,不觉到了梅城。梅城是洢水环抱,双塔对峙,风光十分秀丽。

毛泽东查阅了安化县志。据县志记载:梅城始建于宋代,是一座古老的山城。他和萧子升到东华山看了农民起义烈士墓,调查了清代黄国旭领导的农民起义;走访了一些贫苦农家;又到安化县劝学所拜访了饱学先生、劝学所所长夏默庵。

夏默庵是安化县羊角乡大岩村人,早年毕业于清代两湖贡院,学识渊博,经、史皆通,并善于吟诗作对,一生著有《中华六族同胞考说》、《默庵诗存》、《安化诗抄》等。他晚年曾任安化县教育会长;1917年,64岁的他又担任了安化县劝学所所长。

毛泽东在学校时就曾向安化同学罗驭雄打听过,安化有无什么宿学之士?罗驭雄便向他介绍了夏默庵,所以毛泽东在安化游历了一番之后,便慕名前往拜访。他和萧子升来到夏宅,求见夏老先生。夏家门人通报主人说:有两个年轻的游学先生到访。夏默庵老先生性情高傲,一向不愿意理睬那些游学之士。他叫门人回复说,他不在家。毛泽东二次来到夏宅,夏默庵又以同样的托辞回避了。

毛泽东两次来访都遭到拒绝,可他并不灰心,又第三次登门求见。夏老先生心想:平日里那些游学先生一次遭拒便扬长而去,而这位年轻的游学者究竟有何与众不同之处,我倒要探探他的学问如何。

夏默庵想罢,挥笔写下了半副对联,置于书案之上,方才命门人开门相请。毛泽东进了门,见书案上放着半副墨迹未干的对联,显然是老先生刚刚写下的,便朝老先生躬身问安,尔后方趋至案前看那半副对联,只见上面写的是:

“绿杨枝上鸟声声,春到也,春去也;”

毛泽东看了已知其意,心中暗笑:原来是要考我呀!他略一思索,也不客气,挥笔写出了下联:

“清水池中蛙句句,为公乎,为私乎?”

夏老先生看着毛泽东书写完毕,不觉大吃一惊:这年轻人的对句不但远远胜过了自己的出句,而且还带有火辣辣的批评味道哩!他连声称赞道:

“写得好,写得好!”

夏默庵马上笑脸相迎,邀请毛泽东和萧子升和他一起用餐,并吩咐家人为他们安排了住处。是日晚,夏默庵和毛泽东谈话十分融洽,二人竟成了忘年之交。

第二天,夏默庵赠送毛泽东8块银洋做旅费,并亲自送客人到大门口,依依惜别。

毛泽东和萧子升又在梅城游览了孔圣庙、培英堂、东华阁、北宝塔等古迹,观赏了祭孔用的“铜壶滴漏”。

据萧子升回忆说,毛泽东在梅城曾经写过一首诗,留下来的文字可惜缺了5、6两句,只剩下了:
\begin{couplet}

骤雨东风过远湾,滂然遥接石龙关。

野渡苍松横古木,断桥流水动连环。

……

客行此去遵何路?坐眺长亭意转闲。
\end{couplet}

毛泽东和萧子升还给城中的“鼎升泰”、“谦益吉”、“云祥吉”等店铺送了对联,对方就给几个钱作为路上的费用。

他俩离开梅城后经仙溪、山口、长塘、马迹塘、桃花江,到达了洞庭湖畔的益阳县城。

他们一路上渴了就讨口水喝,或在路旁喝几捧冷水;累了就到池塘里洗澡;有时还在野外露宿。

毛泽东和萧子升在益阳县城游览了市容,走访了一些学校和人士。傍晚时分,他们找到了一家小客栈,客栈里没有别的客人,二人吃了晚饭,就打算在这里过夜。客栈主人是一位20岁上下的漂亮女子,过来收拾罢碗筷,一边抹着桌子一边搭讪道:

“二位先生从哪里来呀?”

毛泽东说:

“我们从梅城来。”

“二位不是本地口音呢。”

“我们是湘潭县和湘乡县人。”

“哎呀,那地方远着呢。”

“大概有1000里路。”

女子又问:

“两位先生要到哪里去?”

“没有目的地。”

她表示不相信。毛泽东说:

“我们是乞丐,所以随便走走。”

女子闻言一愣,显出很吃惊的样子,瞬间又开心地笑了起来,说:

“你们是乞丐?不可能!你们这样斯文,能是乞丐吗?”

萧子升说:

“我们并没有骗你,我们从长沙一路走过来,像乞丐一样。”

她还是感到莫名其妙。毛泽东说:

“你为什么不相信我们的话呢?” 

“因为你们一点也不像乞丐。”

萧子升说:

“乞丐有特别的样子吗?你怎么看我们不像呢?”

女子又仔仔细细端详着两人,说:

“我知道二位都是了不起的人物。我懂得一点看相术,也会测学,可以预知吉凶。这是我爷爷教我的。我爷爷是个诗人,出过一本诗集,叫《桃园曲》。我父亲也是个有学问的人。他们俩在3年时间里相继去世了,只剩下我和母亲在这世上相依为命,为了活命,就开了这个小店。”

萧子升说:

“那你还没有出嫁吗?你能借我看一下你祖父的诗集吗?你无疑也是一位有学问的人。”

“我跟着父亲读了七八年的书,正要开始学写诗的时候,他去世了。我祖父的《桃园曲》收藏在箱子里,明天我找给你们看。”

毛泽东问道:

“你说你会相面,可以给我们看一看吗?”

女子犹豫了一下,说:

“可以是可以,不过说错了二位不要生气。”

只听女子的母亲在里屋那边说道:

“茹英,不要胡闹!不怕得罪客人么?说点别的吧。”

毛泽东却对女子说:

“不要紧,没有关系的。请你直言,看到什么说什么,我们绝对不会生气的。”

女子见毛泽东如此说话,便认真地引经据典,滔滔不绝,把二人今后几十年的功名利禄、婚丧嫁娶、福寿吉凶一一道来。毛泽东和萧子升虽然并没有太在意她所说的那些话,只当是开开玩笑,但见她娓娓道来,如此健谈,倒也觉得十分有趣。女子说完了,又提出了他们为什么要做乞丐的话头。毛泽东就以实言相告,没想到女子竟说她对这种做法非常感兴趣,如果不是家有老母需要服侍,她也打算这样做呢!

第二天早晨,毛泽东二人吃过早饭就要离开了,女子说要他二人再住一日。毛泽东要给她食宿钱,她坚辞不受,问她姓名,她说叫胡茹英。萧子升说:

“日后毛先生要是发达了,他会写信来请你做参谋的。”

胡茹英闻听此言,咯咯笑了起来,说:

“到那时,他也许早已忘记我了。”

毛泽东和萧子升告别胡茹英母女,离开益阳前往沅江。他们走了3个小时的路程,才渐渐看到沅江县城了,可走近一看,不觉大吃一惊,这县城周围到处都是洪水。一个酒店老板告诉他们说:长江发源于高原,每到夏季,高原上冰雪消融,洪水就从西面汹涌而来。因这里地势低洼,所以县城里的街道很快就被淹没了。再过几天,这座县城就要与外界隔绝了。

毛泽东和萧子升见不能继续前行了,就决定提前结束这次游学活动,搭乘民船返回长沙去。二人上了一艘民船,见船上已挤满了乘客,人声嘈杂。毛泽东和萧子升好不容易在一个角落里找到了两个位置。他们刚刚坐下来,就看见前面有两个乘客突然打起架来。这两个人看样子都在50上下年纪,一个脸上刮得很干净,戴副眼镜;另一个是络腮胡子。两人穿着都比较讲究,分明都是体面人,也听不清楚他们嚷嚷的地方方言是什么意思,只见那络腮胡子一把扯下对方的眼镜,狠狠地摔在船板上,他似乎还不解气,又用脚把它踢到河里;失掉眼镜的那汉子,也狠命地撕扯着络腮胡子的袍子,居然被他撕成了两半。

风波终于平息了,络腮胡子把破袍子围在身上,走到毛泽东、萧子升放东西的角落坐下来。萧子升问道:

“怎么回事,那个人为什么撕你的袍子?”

“这个恶棍!”络腮胡子依然满脸怒气:“没有把他扔到河里去,算他运气!”

“他什么地方得罪你了?”

“真是个无赖!”络腮胡子愤愤骂道:“这个家伙过来找地方坐,我给他挪了个地方,让他坐在我的右边,他似乎很感激,自称是常德县衙门的文书。我把我买的两包香烟放在右手边上,不一会儿,等我要抽烟的时候,香烟却不见了。这时他手里拿了一包香烟,正抽出一支,另一包在他口袋里。我看得很清楚,因为他的口袋也不深,他坐下时手里和口袋里并没有东西,而且我的香烟牌子也少见。不用说,肯定是他偷了我的烟。我问他:我的烟呢,他倒对我大喊大叫起来,后来我们就打起来了。这家伙不知道我是沅江衙门的捕快,抓这种小偷是易如反掌的事。”

“好了,好了,别再生气了。”萧子升劝慰他说:“事情过去就算了嘛!”

毛泽东一直没有插话,当听到这个人说他是捕快时,也只是微微冷笑了一下。萧子升望望毛泽东,问道:

“润之,你怎么看这二人打架?一个是捕快,一个是文书,都不是挨饿的人,你看他们都穿得很好的嘛!”

毛泽东叹了口气,摇摇头,依然什么也没有说。

8月16日,毛泽东和萧子升风尘仆仆地回到了长沙。

他们这次游学历时一个多月,途经长沙、宁乡、安化、益阳、沅江5县城乡,步行近千里,所到之处都受到了农民们的欢迎和款待,人家不要一个铜板,就给他们提供吃的喝的,还给他们找地方睡觉。

在这一个多月的奔波中,毛泽东每天清晨早起做过“六段体操”,就追记前一天的经历和心得。一师的师友们传阅了他的游记后,都称赞他是“身无半文,心忧天下”。

毛泽东通过游学,广泛了解了中国社会,了解了农村的现实状况,学到了许多书本上得不到的知识。他认为这是在读“无字之书”。自此之后,毛泽东常常主张走出校门,去读“无字之书”。他把社会看作是人生更重要的一个大学校。

这正是:\begin{xemph}身无半文忧天下,游学胜过读死书。\end{xemph}

欲知毛泽东后来还要在哪些方面进行历练,请看下一章内容。

东方翁曰:毛泽东一人游学或结伴游学,可谓得益匪浅矣。他们这种独特的个体的人生经历,已经成为那个时代的一道独特的社会风景线了。假如能有成千上百万的青年学子同时外出游学,那情形又会是何等的壮观呢?读者诸君知否?这种壮观景象已经不是假设了!就在毛泽东游学50年之后,一次千百万学子遍及全国各地的革命大串联,还真的出现在了世界东方的地平线上:南下北上如穿梭,遍地红旗遍地歌!这次大串联正是在那特定环境下的一次空前规模的“游学”演练,而这场演练的导演和指挥者不是别人,正是具有浪漫主义诗人气质的政治家、思想家毛泽东自己!这一壮举只有亲历者才知道它那深远的历史意义!这一壮举才应该说是真正的开天辟地第一遭——史无前例!
\end{document}
