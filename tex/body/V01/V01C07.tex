\documentclass[../../dazhuan.tex]{subfiles}
% 第一卷
\begin{document}
\chapter*{第七章}
\pdfbookmark{第七章}{V01C07}
\begin{pref}
	吾国学制,课程密如牛毛,虽成年之人,顽强之身,犹莫能举,	
况未成年者乎?况弱者乎?

国力苶弱,武风不振,民族之体质日趋轻细,此甚可忧之现象也。
\end{pref}

话说在1916年暑假期间的一天,毛泽东步行60公里来到板仓杨先生家中。这是他第一次到板仓看望先生。

杨昌济先生的老家板仓在长沙城北面,那里是一个丘陵地带,周围青山环抱,其中有两座山,一个叫影珠,一个叫飘峰,这两座山遥遥相对,板仓就是夹在这两座山之间的一个偏僻山冲。这板仓还不是一个山村,它分为上下板仓,杨昌济先生的家就在下板仓屋。

毛泽东在板仓期间,以极大的兴趣浏览了杨先生的藏书,特别是杨先生订阅的那些新书报刊。他一边看,还一边和先生探讨一些学术问题。

有一次,杨先生在和毛泽东闲谈中,说到距离板仓20多公里的地方住着一位从日本留学归来的柳午亭先生,说柳午亭先生是一位体育运动的热心倡导者和实践者。

毛泽东这一时期正在悉心研究体育问题,一听说附近竟有体育方面的高人,就按捺不住了,第二天就让一位农民带路,去拜访了柳午亭先生。这位柳午亭先生不是别人,他就是后来人们熟悉的无产阶级革命家柳直荀的父亲。

柳先生非常高兴地接待了毛泽东,他们两个人就体育方面的问题进行了广泛交谈。

毛泽东一回到板仓,就同杨昌济先生谈起了此行的印象,他称赞柳午亭先生在体育研究和实践上有很高的造诣,有许多地方是值得效法的。

毛泽东不但善于结交长者和同龄朋友,而且还善于结交年龄比他小得多的朋友,任弼时就是其中的一个。

1916年秋期,毛泽东在学生阅报室里首次认识了任弼时。

任弼时,原名任培国,1904年出生于湖南省湘阴县,1916年秋考入湖南第一师范附属小学。入学的新学期,他在一师学生自治会组织的讲演会上聆听了毛泽东所作的《反对列强瓜分中国》的演讲,使他对毛泽东产生了好奇心和敬仰之情。

在阅报室里,任弼时主动和毛泽东攀谈起来,毛泽东向他介绍了学生自治会的情况,勉励他好好学习,锻炼身体,要能文能武,为改造中国和世界的伟大事业作出贡献。

自此以后,任弼时学习更加勤奋,经常到阅报室和图书室读严复、李大钊等人的著作,还多次参加毛泽东等人组织的学习和讨论。

1916年冬的一天,毛泽东约罗章龙去游岳麓山。二人自清晨8时启程出南门,冒着严寒,准备从朱张渡过湘江。

朱张渡,因朱熹和张南轩而得名。朱熹、张南轩都是宋代有名的学者和诗人,还都在长沙讲过学。湖南知识青年对他们的故事都很熟悉。

毛泽东和罗章龙在朱张渡口的茶亭里坐了下来,他们讨论了朱熹、张南轩在湖南所留下来的思想影响。尔后他们就去攀登岳麓山的麓宫,傍晚时分才下山。在过赫曦台时,毛泽东看到了朱熹和张南轩的联句,忽有所感,吟成了《五律》一首,诗云:
\begin{poem}
共泛朱张渡,层冰涨橘汀。鸟啼枫径寂,木落翠微冥。

攀险呼俦侣,盘空识健翎。赫曦联韵在,千载德犹馨。
\end{poem}

1916年12月9日,毛泽东给黎锦熙写了一封信,他写道:
\begin{xquote}
邵西仁兄大人阁下:

去冬曾上一函,所言多不是,得书解责,中心服之。前之所言,诚自知其不当。袁氏笼络名士,如王、梁、章、樊诸人,均坠其术中,以此联想及兄。其实兄尚非今所谓名士也。事务之官,固不同乘权借势之选,而兄之所处,不过编书,犹是书生事业,并事务官而无之,于进退之义何有?此弟之甚妄言也。辱教:学宜自造,不必因人;心情求全而去偏。此诸义者,皆书诸绅矣。又嘱以常常通信,心中无所见,有之矣,又以为不足质诸左右,增笔墨裁答之劳。今夏阅报,见兄“以国语易国文”一文,私意不尽谓然,拟发所见,以资商榷。又念于此道并无研究,一隅之见,自以为是者,未必果是,为此而止。今乃有进者:古称三达德,智、仁与勇并举。今之教育学者以为可配德智体之三言。诚以德智所寄,不外于身;智仁体也,非勇无以为用。且观自来不永寿者,未必其数之本短也,或以其身体之弱然也。颜子则早夭矣;贾生,王佐之才,死之年才三十三耳;王勃、卢照邻或早死,或坐废。此皆有甚高之德与智,一旦身不存,德智则随之而隳矣!夫人之一生,所乐所事,夫曰实现。世界之外有本体,血肉虽死,心灵不死,不在寿命之长短,而在成功之多寡。此其言固也。然苟身之不全,则先已不足自乐于心,本实先拔矣。反观世事,何者可欣?观卢升之集,而知其痛心之极矣。昔者圣人之自卫其生也,鱼馁肉败不食,《乡党》一篇载之详矣。孟子曰:知命者不立夫岩之下。有身而不能自强,可以自强而故暴弃之,此食馁败而立岩墙也,可惜孰甚焉!兄之德智美矣,惟身体健康一层,不免少缺。弟意宜勤加运动之功。弟身亦不强,近以运动之故,受益颇多。闻之至弱之人,可以进于至强。东西大体育家,若罗斯福,若孙棠,若嘉纳,皆以至弱之身,而得至强之效。弟始闻体魄、精神不能并完,且官骸肌络及时而定,不复再可改易,今乃知其不然。心身可以并完也,而官骸亦无时不可改易也。愚意如此,不知合兄之心否?余不多言,敬请

教安!

\begin{sign}小弟 泽东谨上\end{sign}
\end{xquote}
毛泽东在信中所说的“身心可以并完”,是他坚持体育锻炼的亲身感受。长期以来,他为了强壮身体,磨砺意志,同时也是为了保持学习的旺盛精力,一直坚持着体育锻炼。

第一师范前面那条宽阔的湘江就是毛泽东常去游泳的地方。为了结伴游泳,他曾经贴出过一张启事,上面写着:

铁路之旁兮,水面汪洋。深浅合度兮,生命无妨。凡我同志兮,携手同行。晚餐之后兮,游泳一场……

同学们看了这则启事,纷纷报名。毛泽东很快就组织了一支100多人的游泳队伍,其中便有蔡和森、张昆弟、罗学瓒等人,他们经常和毛泽东一起来到湘江游泳、畅谈。罗学瓒在毛泽东影响下,潜心学习游泳,逐步认识到了游泳的好处。小小年纪的杨开慧也感到很好奇,就总是跟着这群大哥哥们到江边观看。

毛泽东不但可以横渡湘江,他还常常从猴子石游到相距几华里的牌楼口。在那深秋时节,万山红遍,岳麓山下的枫林披上了嫣红的新装。湘江的碧波,更加清澈晶莹。江面上千帆竞发,百舸争流。这正是毛泽东游泳的大好时节,他和同学们时而向前劈波斩浪,时而仰卧在水面上,观察着天空中那展翅飞翔搏击苍穹的雄鹰。此时的毛泽东,身心充满了力量和信心。就是在那寒风刺骨的冬季,很多同学不敢下水了,毛泽东也毅然坚持到江中去游上一番。

1917年3月,日本友人宫崎寅藏来到了长沙城,专程参加黄兴的葬礼。

宫崎寅藏自号白浪滔天,他曾经积极支持和帮助孙中山领导的中国资产阶级民主革命,与黄兴等人有着很深的交往。

毛泽东获悉宫崎寅藏来长沙的消息后,邀请1916年毕业后在外地教小学的萧子暲一起联名写信给宫崎寅藏,请求宫崎寅藏当面赐教。他在这封信中写道:
\begin{xquote}
白浪滔天先生阁下:

久钦高谊,觌面无缘,远道闻风,令人兴起。先生之于黄公,生以精神助之,死以涕泪吊之;今将葬矣,波涛万里,又复临穴送棺。高谊贯于日月,精诚恸乎鬼神,此天下所希闻,古今所未有也。植蕃、泽东,湘之学生,尝读诗书,颇立志气。今者愿一睹丰采,聆取宏教。惟先生实赐容接,幸甚,幸甚!

\begin{sign}
湖南省立第一师范学校学生  萧植蕃 毛泽东 上\end{sign}
\end{xquote}

毛泽东执笔的这一件书迹,宫崎寅藏十分珍惜,他和他的后人相继收藏至今。

1917年4月1日,毛泽东以“二十八画生”为笔名所撰写的一篇长达7000余字的《体育之研究》,经杨昌济先生推荐,陈独秀将其发表在《新青年》第3卷第2号上。

这篇长文是毛泽东在体育锻炼的同时,对体育理论进行探索性研究的重要成果。《体育之研究》开头有一个前言,正文分为8个部分:第一、释体育;第二、体育在吾人之位置;第三、前此体育之弊及吾人自处之道;第四、体育之效;第五、不好运动之原因;第六、运动之方法贵少;第七、运动应注意之项;第八、运动一得之商榷。

毛泽东在前言中写道:

“国力苶弱,武风不振,民族之体质日趋轻细,此甚可忧之现象也。体不坚实,则见兵而畏之,何有于命中,何有于致远?坚实在于锻炼,锻炼在于自觉。”

“善其身无过于体育。体育一道,配德育与智育,而德智皆寄于体,无体是无德智也。体者,为知识之载而为道德之寓也;体育于吾人实占第一之位置,体强壮而后学问道德之进修勇而收效远;近人有言曰,文明其精神,野蛮其体魄,此言是也;欲文明精神,先自野蛮其体魄;苟野蛮其体魄矣,则文明之精神随之。意志也者,固人生事业之先驱也;运动而有恒,第一能生兴味,第二能生快乐。”

接下来,他对当时学校不重视体育的现象提出了批评: 

“吾国学制,课程密如牛毛,虽成年之人,顽强之身,犹莫能举,况未成年者乎?况弱者乎?观其意,教者若特设此繁重之课以困学生,蹂躏其身而残贼其生,有不受者则罚之。”

毛泽东还认为,体育关系着卫国力量的强弱:

“动以营生也,此浅言之也;动以卫国也,此大言之也。”

他还写道:

“夫体育之主旨,武勇也。武勇之目,若猛烈,若不畏,若敢为,若耐久,皆意志之事。取例明之,如冷水浴足以练习猛烈与不畏,又足以练习敢为。凡各种之运动持续不改,皆有练习耐久之益,若长距离之赛跑,于耐久之练习尤著。夫力拔山气盖世,猛烈而已;不斩楼兰誓不还,不畏而已;化家为国,敢为而已;八年于外,三过其门而不入,耐久而已。要皆可于日常体育之小基之。意志也者,固人生事业之先驱也。”

“三育并重,然昔之学者详德而略于体。”“偻身俯首,纤纤素手,登山则气迫,涉水则足痉。故有颜子而短命,有贾生而早夭,王勃、卢照邻,或幼伤,或坐废。”

毛泽东分析了“今之学者多不好运动”的原因:

一是“无自觉心”。“今日不为,他日将无以谋生。”二是“积习难返”。“我国历来重文,动有‘好汉不当兵’之语。”三是“提倡不力”。四是“以运动为可羞”。“忽尔张臂露足,伸肢屈体,此何为者邪?宁非大可怪者邪?深知身体不可不运动,且甚思实行,竟不能实行者;有群行群止能运动,单独行动则不能者;有燕居私室能运动,稠人广众则不能者。”

他总结了自己从事体育运动的经验,认为体育运动“不重言谈,重在实行。”他提出“运动所宜注意者三:有恒,一也;注全力,二也;蛮拙,三也。”“运动既久,成效大著,发生自己价值之念。以之为学则胜任愉快,以之修德则日起有功,心中无限快乐。”

毛泽东在文章的最后一部分“运动之一得之商榷”中,介绍了他创造的“六段运动”。这6段是:手部、足部、躯干部、头部、打击运动,调和运动。

在杨昌济的学生中,毛泽东年龄比较大,思想也比较激烈深刻。因此他留给杨开慧的印象也最深。杨开慧秀外慧中,举止温文尔雅,性格坚强,喜沉思,有理想,有追求。她非常钦佩毛泽东的抱负和见解。有一次,她在一篇作文中写道:“要救国就要锻炼强健的身体。”这句话正是毛泽东在《体育之研究》中所阐述的中心思想。

毛泽东的学问品行也同样吸引着杨开慧。每当毛泽东等人来向父亲请教时,她总是搬一条小凳,静静地坐在一边,成为一名沉默而热心的听众。一开始,她还只是默默地听,听他们谈论治学、做人之道,听他们研讨朝代兴衰,听他们探寻救国救民真理。

    杨昌济对这个聪明的女儿很是看重,对她的介入也极为赞许,并不时向毛泽东等人推荐女儿的学问。时间一长,杨开慧也逐渐加入到了他们的讨论之中,自然成了这批学生中的一员,和他们一起议论时事,抨击时政,并和毛泽东等相互传阅笔记,交流心得体会。频繁地接触,使杨开慧情不自禁地萌发了对毛泽东的倾心爱慕之情。毛泽东也像兄长一样喜欢比他小8岁的小师妹。

杨开慧除了向毛泽东学习一些思想方法外,还慢慢地接受了他的生活方式和锻炼方法,坚持进行深呼吸,常吃硬的食物和洗冷水浴等等。

说起这洗冷水浴,原是杨昌济先生常年进行的锻炼项目。毛泽东一开始只是模仿先生,他遵照先生的嘱咐,先从秋天开始,使皮肤逐渐适应冷水的刺激,这样从秋到冬,就可以一直坚持下去了。

一师学生宿舍旁边有一口清凉的水井,每天清晨天蒙蒙亮,许多同学还在睡梦中,毛泽东就起了床,带着罗布浴巾,来到水井旁,在这里进行冷水浴。他脱光上身衣服,吊上一桶桶的井水,往身上浇,然后擦,擦了又浇,浇了又擦,一遍遍地浇着,欢呼着,一遍遍地擦着,舒展着,这样反复一二十分钟,直到全身发红发热为止。他洗完了冷水浴,这才穿好衣服进教室自习。同学们问他洗冷水浴有什么好处,他说:

“冷水浴好处可多哩,最主要它有两大好处:1、可以促进血液循环,增强身体抵抗力,并能强壮筋骨;2、可以培养勇猛无畏的气魄和战胜困难的精神。”

有同学问:

“洗冷水浴的好处虽多,但在冬天搞冷水浴实在太难受了,你是不是一开始就不感觉到难受呢?”

毛泽东回答说:

“最初是感觉难受的,但只要下定决心,突破了这个难关,也就不那么冷了。任何一项体育活动,要把它坚持到底,都不是很容易的事情,关键在于一个人有没有决心和毅力。只要有决心和毅力,就会由勉强到不太勉强,由不太勉强到不勉强,就会坚持到最后,就可以习惯成自然,不会感到有什么困难了。所以重要的问题在于持之以恒。”

在一师进行冷水浴锻炼的学生起初只有毛泽东、蔡和森、罗学瓒几个人,后来逐渐发展到了20多人。大家光着膀子,只穿裤衩,各人从井里提一桶凉水,一声令下,从头淋遍全身。有时他们还互相对着淋。欢声、笑语,伴着这泼水声,显示着青年人的勃勃生机。

杨开慧正是受了父亲和毛泽东等人的影响,便也开始常年坚持冷水浴了。

且说1917年4月16日,黎锦熙自北京回到长沙。他在日记里有这样的记载:“上午到社(指宏文图书社——笔者注)晤毛润之,谈学。”

1917年6月间,湖南一师在学生中开展了一次“人物互选”活动,目的是考察他们的学业与操行,促进他们积极向上,培养更多的人才。

据1918年编印的一师校志记载:这次互选的标准包括德育、智育、体育三个方面。其中德育又分为敦品、自治、好学、克俭、俭朴、服务等项;智育又有文学、科学、美感、职业、才具、言语等项;体育又分为胆识、卫生、体操、国技、竞技等项。互选的方法是每个学生只能投3票,每张票上只能写1个人,且被选人不以同级和同班同学为限;投票采用双记名法,被选人的名字写在选票上端之右,选举人的名字写在选票下端之左;选举人还要按照互选的标准,将被选举人的一些事例及自己对被选人的评语详注在票内。

检票的结果是,全校11个班有400余名学生参加了“人物互选”,当选者有34人。其中毛泽东在德、智、体三方面的6项中,“敦品”得11票,“自治”得5票,“文学”得9票,“言语”得12票。“才具”得6票,“胆识”得6票,共计49票,为全校之冠。周世钊以47票名列第二;张昆弟名列第四。

在这当选的34人中,在德、智、体三方面都有项目得票者仅有毛泽东一人。毛泽东成为一师最优秀的学生。他的全面发展,他的进步言论、高尚道德、过人胆识、进取精神和超群才智,无不受到同学们的钦佩和推崇。     

1917年盛夏,毛泽东和罗学瓒等同学一起在湘江中游泳。时江水大涨,几死者数。毛泽东在水中迎风劈浪,欢畅淋漓,豪气顿生,不禁吟道:

“自信人生二百年,会当水击三千里”。

与一师隔江相望的岳麓山,还有江中的水陆洲、牛头洲,是毛泽东常去锻炼身体的地方。他除了爬山外,还在这些地方进行雨浴、日光浴、风浴、雪浴等活动。

在这炎炎的夏日里,在大雨倾盆之时,毛泽东常常脱去衣服,冒雨跑步,这叫作“雨浴”。在烈日当空之时,他总是脱去衬衫,穿一条短裤,赤膊在太阳底下走来走去,任凭阳光直射;有时也会在江中游泳后走上水陆洲、牛头洲,躺在沙滩上,让太阳晒遍全身,这叫作“日光浴”。而在那寒冷的冬季里,他也常常脱去棉衣,穿着单薄的衣服,或在山麓,或在山峰,或在江边,迎着寒风大声呼叫,做跑跳运动,任凭寒风侵袭,这叫作“风浴”。在那大雪纷飞的日子里,一团团、一簇簇的白雪,铺天盖地,他却站在雪地里,一任自己的思绪神驰遐想,这叫作“雪浴”。

有的同学对毛泽东的这些活动很不理解,觉得他的行动有些怪异,便在背后批评议论他。毛泽东可不管这些,依然是我行我素。

长沙周家台子蔡和森的家“沩痴寄庐”,是毛泽东和同学们经常聚会交流学习心得、讨论时政大事的一个场所。毛泽东到蔡家就像到了自己的家一样,一来就先到菜园拔草、浇水,然后一起吃饭,有时讨论起来通宵达旦。

有一次,在一个狂风暴雨、电闪雷鸣的晚上,毛泽东独自一人顶风冒雨爬上岳麓山顶,然后又从山顶跑下,来到了蔡和森家。蔡和森的母亲葛健豪老人问他是怎么回事,他回答说,是为了体验《尚书·舜典》(一说《尚书·尧典》)上所说的“纳于大麓,烈风雷雨弗迷”这句话的意境,锻炼自己的意志和胆量。

葛健豪,原名兰英,1865年出生于湖南湘乡县荷叶村。1911年辛亥革命爆发,她不顾族人反对,变卖了嫁妆,带领儿子蔡和森、女儿蔡畅外出求学。1913年,葛健豪高小毕业后,回乡创办一所简易小学,自任校长。不久,她又进入湖南女子教员养习所,1915年毕业后,创办了湘乡第二女子简易职业学校,自任校长。

且说1917年10月8日,罗学瓒在日记中记载了关于洗冷水浴的一些情况,他写道:

“余前数日,因浴冷水,致身痛头昏。休养数日,少饮食,多运动,今日已痊愈,复与毛君泽东等往河边洗擦身体一番,大好快畅。”

毛泽东除了上述锻炼项目外,还进行野外露宿锻炼。一师校园里的君子亭和岳麓山的爱晚亭、白鹤泉以及橘子洲等地,是他经常露宿的地方。

在寒霜时节,当夜幕降临时,他就邀请一些同学来到野外高谈阔论,直到夜深人静,然后各自找个地方,露宿至天明。

有一天清晨,几个游人来到岳麓山,见庙旁露天底下的一张长板凳上睡着一个人,头脚都用报纸盖着,他们就好奇地凑过去察看。睡着的那个人听到游人的脚步声,翻身起来,收拾好报纸就走了。此人不是别人,正是毛泽东。

毛泽东有半年时间不在宿舍里就寝,还有好长一段时间只吃一顿饭,故意饿肚子。

1936年,他在同斯诺谈话时对在一师的体育锻炼作了回忆,他说:“我们也热心于体育锻炼。在寒假中,我们徒步穿村越林,爬山绕城,渡江过河。遇见下雨,我们就脱掉衬衣让雨淋,说这是雨浴。烈日当空,我们也脱掉衬衣,说是日光浴。春风吹来的时候,我们高声叫嚷,迎着狂风朗读唐诗,说这是叫作‘风浴’的体育新项目。在已经下霜的日子,我们就露天睡觉,甚至到十一月份,我们还在寒冷的河水里游泳。这一切都是在‘体格锻炼’的名义下进行的。这对于增强我的体格大概很有帮助,我后来在华南多次往返行军中,从江西到西北的长征中,特别需要这样的体格。”

这正是:\begin{xemph}奔走呼号卧霜野,特立独行骇俗尘。\end{xemph}


浪里击水发浩歌,练就惊天动地身。

欲知毛泽东还有哪些非常之举,请看下回分解。

东方翁曰:毛泽东1916年在给他的老师黎锦熙的信中,论述了德、智、体三者之间的关系,立意甚高。他旁证以“东西大体育家若罗斯福,若孙棠,若嘉纳,皆以至弱之身,而得至强之效。”使德、智、体三育并重的说服力愈发感人了。毛泽东之对于体育,研究既专,心得必多,于是乎,一篇由体育论及民族兴衰、国力强弱、教育积弊的洋洋洒洒的七千言《体育之研究》,便借《新青年》这一载体问世了;其文采、其例证、其深意,今日读来,亦令人觉得是中国体育史上殊为难得的一篇大作。
\end{document}
