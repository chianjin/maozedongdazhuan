\documentclass[../../dazhuan.tex]{subfiles}
% 第一卷
\begin{document}
\chapter*{第一章}
\pdfbookmark{第一章}{V01C01}

\begin{pref}
大宋天子赵匡胤说过:有钱龟孙不讲理!
\end{pref}

话说在中国湖南省中部丘陵地带有一座山,名叫韶山。韶山的顶峰就是著名的韶峰,又称仙女峰,它是南岳衡山72峰之中的第71峰,也是韶山八景之一。韶峰海拔500多米,山高陡峭,气势雄伟。站在韶峰之上,给人的感觉是:“绝顶才宽三五尺,此身如在九重天。”有诗赞曰:

绕岫岚光凝欲滴,长风轻袅云烟侧。山涵五月六月寒,地拥千山万山碧。

从来仙境称韶峰,笔削三山插天空。天下名山三百六,此是湘南第一龙。

就在这韶峰下,群山环抱着一块不大的谷地,走向是由南向北,长5公里,宽3.5公里。地形明显分为东西两部,西部山峦环绕,东部丘岗起伏,大致构成“六山一水二分田,一分道路和庄园”的格局。这个山谷就叫“韶山冲”。

小小的韶山冲,隶属湖南省湘潭县管辖,在湘潭、宁乡、湘乡三县交界处。它东北距湖南省城长沙90公里,东南离湘潭县城35公里,是个偏僻的小地方。在中华人民共和国成立以前,这里既没有铁路,也没有公路,更没有通航的河道。

韶山冲里居住着毛、李、钟、周、邹、彭、庞几姓人家,他们多是忠厚、朴实、勤劳、善良的农民,在帝国主义和封建主义的压迫下,家家都遭受着深重的灾难,过着贫困的生活。韶山冲的农民们世世代代流传着这样的歌谣:

其一:韶山冲,长又长,砍柴做工度时光。鸡鸣未晓车声叫,隔夜难存半合粮。

其二:韶山冲,冲连冲,十户人家九户穷。有女莫嫁韶山冲,红薯柴棍度一生。

其三:农民头上三把刀,税多、租重、利息高;

农民眼前路三条,逃荒、讨米、坐监牢。

在这韶山冲里流淌着一条小溪,终年潺潺不断。小溪上游的南岸,有一栋依山傍水的半瓦半茅的房屋,叫上屋场。这是一栋湖南农村常见的“凹”字形住宅,当地人称作“一担柴”式的房子。这栋房子从堂屋正中分界,两边各住着一户人家,东边瓦房里的这一户人家是毛氏家族。

说起韶山冲毛氏家族的历史,可以往上追溯到14世纪中叶。那时候,元末红巾军起义正如火如荼,天下大乱。在这个王朝更迭的年代,江西省吉州府龙城县,也就是现在的江西省吉水县,有一个名叫毛太华的青年农民,不甘老死蓬蒿、寂寞一生,于是便毅然弃农从戎,投奔到朱元璋的农民起义军中去了。

朱元璋建立明朝以后,毛太华做了一个百夫长一类的下级军官,并随从明朝大将傅友德和蓝玉远征云南。在云南归入大明的版图后,毛太华被留了下来,成为一名戍滇军官。

云南边陲为少数民族聚居地,当时很少有汉民居住。在云南澜沧,毛太华同许多戍滇军人一样,娶了一位当地少数民族姑娘为妻,生育了4个儿子,依次取名为:毛清一、毛清二、毛清三、毛清四。

毛太华渐渐衰老了,他念念不忘内地的故乡,于是就告老还乡。因为他立有边功,就被准许迁回内地。明朝洪武十三(1380)年,毛太华偕妻子王氏及长子毛清一、四子毛清四回到了湖南,居住在湘乡县城北门外的绯紫桥,购买了田产几十亩。

数年后,毛清一、毛清四迁移到湘潭县七都七甲韶山定居。自此,毛氏家族便在这山青水秀的韶山东茅塘一带繁衍生息,毛太华便成为韶山毛氏家族的第一代祖先。

毛太华随着儿子毛清一、毛清四在韶山冲生活若干年后就去世了。毛氏家族从毛太华这一代到后来的第6代,都没有固定的谱系。到了清朝乾隆二年(1737年),毛氏家族从第7代才开始修族谱,他们定下的谱系是:

立显荣朝士,文方运际祥;

祖恩贻泽远,世代永承昌。

在毛氏族谱中,记载着颇为严格的家训:培植心田、品行端正、孝养父母、友爱兄弟、和睦乡邻、教训子孙、矜怜孤寡、婚姻随宜、奋志芸窗、勤劳本业。

族谱中规定的家戒是:“戒游荡”、“戒赌博”。

清朝光绪七年(1881年),毛氏族谱再一次修订,续订的谱系为:

孝友传家本,忠良报国光;

启元敦圣学,风雅列明章。

毛氏家族第17代传人叫毛祖人,又名毛四端。毛祖人出生于1823年,他是一个沉默寡言、勤劳忠厚的农民,没有读过书,主要靠种田、出卖劳动力维持一家人的生计。毛祖人生育有两个儿子,长子叫毛德臣,次子叫毛翼臣。

毛德臣,谱名恩农;生育有3个儿子,长子毛菊生,次子毛梅生,三子毛献生。

毛翼臣,谱名恩普,字寅宾,生于1846年5月22日,逝于1904年11月23日。毛翼臣娶妻刘氏,生有一子,取名毛贻昌;他还有两个女儿,成人后分别嫁到了张家与贺家。

1878年,毛翼臣与哥哥毛德臣分家,他从祖居地韶山东茅塘搬到了韶山冲南岸的上屋场。毛翼臣是一个老实厚道的庄稼人,一生清贫,为了生活,不得不将祖传的部分田产典当出去。

毛翼臣的儿子毛贻昌,字顺生,号良弼,生于1870年10月15日,在族兄弟中排行老大,他读过几年私塾,懂得一些算术。后来他终身务农经商,而且生财有道。

毛顺生10岁那年,由父母做主,与湘乡县四都唐家圫也就是被后人改称为棠佳阁的文芝仪之女、13岁的文七妹订了婚。

唐家圫文家是一个大家族,共有兄弟文芝兰、文芝仪、文芝祥3大房。

文芝仪和妻子贺氏,生育有3子3女。

文芝仪的长子文玉瑞在族兄弟中排行第七,生有长子早夭;次子文涧泉,在族兄弟中排行十一;三子文梅清,在族兄弟中排行十七。

文芝仪的次子文玉钦在族兄弟中排行第八,有一个继室,两个姨太。他生有长子文泮香,排行第十;次子文运昌,排行十六;三子文南松,排行二十;女儿文静纯,就是文运昌的三姐。

文芝仪的三子文玉材,幼年早夭。

文芝仪的长女嫁给了钟姓人家,称钟文氏;次女嫁给了王姓人家,称王文氏,就是后边将要说到的王季范的母亲;三女则许给了毛顺生,她就是1867年2月12日出生的文七妹。

文七妹姊妹几个同当时许多农村妇女一样,没有一个正式的名字。她二姐在文家同辈姊妹中排行第六,她排行第七,所以家里人就分别叫她们“六妹”、“七妹”,外姓人便称她们“文六妹”、“文七妹”。

据毛泽民的外孙曹耘山说,他在苏联的档案中发现了一份毛泽民1939年6月初到苏联以后为毛泽东代填的《个人履历表》,毛泽民在履历表中将他们母亲的名字写作“文素勤”。由此,便有人断定毛泽东的母亲叫文素勤。毛泽民这样写,可能是出于当时的某种考虑或需要,而一些人却据此要推翻毛泽东1936年同斯诺谈话时的说法,是十分不慎重、不妥当的。

且说文家是以务农为业,家境小康。他们所在的唐家圫距韶山冲有10余公里。由于文七妹的曾祖父逝世后葬在了韶山冲龙眼塘,所以后代人每年都要到韶山扫墓、祭拜。文家为了到韶山扫墓时有个落脚的地方,便将文七妹许给了韶山冲的毛顺生。

毛顺生15岁那年和18岁的文七妹完了婚。

此时的文七妹,生得是中等身材,清秀端庄,宽前额,长圆脸庞,浓眉大眼。她初到毛家时,很不习惯,经常回娘家哭脸。因为文家是个四世同堂的大家族,她嫌在毛家太冷清。在家作女时,只做一些家务事,田间劳动都是由男人们去干。她嫁到韶山后,毛家男劳力少,里里外外的活都得干。不过她后来也就慢慢地习惯了。

毛顺生16岁时为了偿还父亲欠下的债务,到湘军里当了近两年兵,把军队发的饷银积攒起来,退伍后还清了债务。17岁时,他开始当家理事。那时家里只有六七亩地,家底微薄,一家人终年为温饱而操劳。但此时的毛顺生已经成为一个魁梧高大、精明能干、善于经营的农民,把一家人省吃俭用节余下来的稻谷进行加工,将白米挑到银田寺赶集出售,有时也零售给附近的穷苦樵夫和手工业者;而那些米糠则用来喂养猪仔出售。

毛顺生治家严谨,是个好当家的。他常说:“吃不穷、用不穷,人无计算一世穷。谁会盘算,谁就能过上好日子;不会盘算的人,你给他明晃晃的金山银山,也是空的。”他还有一个勤俭持家的贤内助文七妹,所以这个贫困的家庭就慢慢富裕起来,渐渐积攒了一笔钱,赎回了父亲典当出去的田产,共有耕地15亩,年收60担谷。

公元1893年12月26日清晨,即清王朝光绪十九年癸巳十一月十九日卯时破晓之际,在上屋场东边姓毛这户人家里,一个伟大的小生命就像天边的一轮红日喷薄而出,呱呱坠地,他就是毛氏家族的第20代传人、毛翼臣的孙子、毛顺生和文七妹的第3个儿子。毛顺生自然是喜上眉梢,按照族谱中的“泽”字辈,为儿子取了一个响亮的名字,叫毛泽东,字润之。

文七妹先后生育了7个孩子,长子、次子均在襁褓中夭亡,这第三胎便是毛泽东。后来她又生下两男两女,四子毛泽民,又名毛泽铭,字咏莲、润莲,1896年4月3日出生,毛泽东称为四弟;第五个儿子毛泽覃,又名毛泽淋,字咏菊、润菊,1905年9月25日出生,在堂兄弟中排行老六,毛泽东称为六弟(毛泽东这一代人堂兄弟共10人,除了夭折的两个兄长,还有毛泽荣行五,老七早年夭折,毛泽华行八,1928年在上海搞地下工作牺牲,毛泽连行九,毛泽青行十);而那两个女儿又都夭折了。

在毛泽东和毛泽民出生后,毛顺生夫妻是上有父亲,下有两个儿子,全家5口人,每年除去口粮35担外,还有25担左右的余粮,日子过得是有滋有味。此时的文七妹也成为家里最忙碌、最辛苦的人了,她既要抚育孩子,又要操持家务,还要养鸡喂猪、种菜除草,样样活都得干,方方面面都需要安排得有条有理。

且说毛泽东的出世,给毛家带来了说不尽的欢喜,可文七妹在欣喜中心里又不免有一点难以抑制的隐忧,她唯恐这个儿子再出什么事,便多方烧香拜佛,祈求神灵保佑,并开始吃起了“观音斋”。尽管如此,她还是不放心,在毛泽东出世两个月后,便把儿子带回了唐家圫的外婆家,托付给外婆抚养。外婆让毛泽东拜七舅父文玉瑞、七舅母赵氏为干爹干娘,有托福之意;因为七舅父七舅母子女众多,被认为是命中多子多福。

在唐家圫的后山,有个龙潭坳,坳内有一股清泉,一年四季水流不断。在龙潭坳旁边矗立着一块巨石,高2.8丈,宽2丈。传说这里曾经出过一个妖怪,常常兴风作浪,后来有个能人为民除害,用这块巨石将妖怪降服了,人们就把这块巨石称为“石观音”,并经常来此祷告祈福。外婆贺氏生怕小外孙“根基不稳”,为了让小外孙能长大成人,就让文七妹抱着毛泽东来到“石观音”前烧香叩头,拜这块巨石为“干娘”,寄名“石头”,取容易抚养之意。因毛泽东排行第三,由此他又有了一个“石三伢子”的乳名。自此以后,他每次从巨石前经过,大人们都要说一声:“石头,给干娘磕个头。”石三伢子便乖巧地朝着“石观音”拜上一拜。

毛泽东童年的大部分时间是在唐家圫外婆家度过的。这是一个四世同堂的大家庭,全家老少20余口人,生活也比较富裕。在一大群孙男孙女中,多了个“根基不稳”的小外孙,外婆自然是对他宠爱有加。毛泽东在这里过着无忧无虑的群体生活,他同表兄弟表姐妹们一块儿嬉戏,一道去拔草,捡柴,拾豆子,放牛,打猪草。

毛泽东每次放牛,都把小伙伴们邀集在一起。他往往把大家分成3组,一组看牛,一组割牛草,一组上山摘野果子。临了,大家把牛栓到树上,每人分得1份牛草,然后就坐在树下吃野果子,讲故事,做各种游戏,最后尽兴而归。

在毛泽东8岁那一年的正月,他跟着母亲看耍狮子,心里一高兴,便脱口念道:“狮子眼鼓鼓,擦菜子煮豆腐,酒放热气烧,肉放烂些煮。”逗得外婆家的人都笑了起来。

1902年,毛泽东9岁了,该上学了,元宵节过后,毛顺生就把他从唐家圫接了回来,送到离家不到200米的南岸私塾读书,接受启蒙教育,并给他取字叫咏芝。

后来毛泽东学富五车,知识渊博,通晓姓名之学,他常常将他的字写作润之或者润芝。再后来,毛泽东在他的一生中,又在不同时期分别用过润、泳之、二十八画生、久滋、子任、润子任、杨子任、石山、学任等等名、字和号,还曾经用过与任、自任、事任、赵东、李德胜、李得生、得胜、国彬、杨先生等等笔名和别名。

且说毛泽东的第一个塾师是邹春培,年约50多岁。在入私塾那天,邹春培身着长衫,把毛泽东领到摆放在东墙下的神龛前面,严肃地说道:

“这是大成至圣文宣王孔夫子的神位,从今天起,你每天早晨进来,都要给孔夫子作揖。以后你就会文思发达,连中三元。”

毛泽东便恭恭敬敬地对着孔夫子作揖行礼。邹先生见状,高兴地对毛顺生说:

“令郎有朝一日,定会名登高科,光宗耀祖。”

毛顺生则有自己的打算,连忙说:

“种田人家的子弟,不稀罕功名利禄,只要算得几笔数,记得几笔账,写得几句来往信札,就要得了。”

他是这样说的,也是这样做的。他一心想把自己的儿子培养教育成为能够继承家业的人,所以在把儿子送进私塾读书的同时,还亲自训练儿子双手打算盘,教儿子记账,要儿子学会经商的本领。

文七妹和韶山的大多数人一样,只字不识;她还和许多人一样,也是个信佛的人。毛泽东在上私塾以前和到私塾念书以后,常常陪着母亲一起到附近的凤凰山寺庙里求神拜佛。可父亲毛顺生却不信佛,这使毛泽东大伤脑筋。9岁那年,他曾经和母亲讨论过父亲不信佛以及如何帮助父亲的问题。

毛泽东曾经对斯诺说过:“我父亲早年和中年都不信神。当时和以后,我们想过好多办法想让他信佛,可是没有成功。他只是咒骂我们,我们被他的攻击所压倒,只好退让,另想新的办法。但他总是不愿意信神。”

可是有一次,毛顺生外出要账,在回家的路上碰到了一只老虎。这只老虎因受惊逃掉了,毛顺生也死里逃生。从此他比较信佛了,有时还要烧烧香。

再说毛泽东同旧时代所有私塾学生一样,先是从《三字经》读起:“人之初,性本善……”;接着开始点读《论语》、《孟子》和《诗经》,由于他天资聪颖,读书又很用心,所以塾师教过的书,他都能背得下来。他自己还学会了使用《康熙字典》,一些先生没有点的书,他也能读懂。在学习上他不需要先生太劳神,因此大家都叫他“省先生”。

毛泽东在读书上更不用父亲操心,但他常常会做出一些“离经叛道”的事情来,这倒叫父亲大伤脑筋。

在南岸私塾后边住有一户人家,女主人叫邹四阿婆。她家房屋旁边种有一些枇杷树,柚子树和桃子树。每到果子成熟的时候,毛泽东和他的小伙伴们的眼睛就被树上的果子给勾住了,恨不得变成黄莺和八哥,飞上去一口一个。只是邹四阿婆看守得严,手里还常常拿着一根长竹竿,随时准备教训偷果子的小孩。哪个小孩偶尔偷了她几个果子,她又是嚷,又是追打,还要向家长和私塾里的先生告状。小孩子们在背后都骂她“小气鬼”、“背时婆”,于是也就存心跟她作对。

有一次,毛泽东和小伙伴们瞅来瞅去不见邹四阿婆的踪影,便相互发了个信号,像猴子一般爬上树去。冷不防吓人的叫骂声突然传来:“好呀!”也不知在哪儿猫着的邹四阿婆拱了出来:“哼,这回看你们‘小泥鳅’往哪里钻!”她冷笑着,挥动竹竿就打。

“跑!”毛泽东一声令下,指挥着伙伴们分头躲闪。邹四阿婆看到他比一般孩子长得高大,又是个“头儿”,便把竹竿转向他,呵斥道:“石三伢子,我先逮住你!”

毛泽东和小伙伴们纵身翻过一道沟,爬上两棵大枞树,等邹四阿婆跌跌撞撞地追来,他们已经高踞树杈,嬉笑着,喘着气做鬼脸。

邹四阿婆还会做干果,什么梅子、黄瓜、茄子一类普普通通的东西,经她的手一摆弄,就变成了酸中有甜、甜中含辣的“山珍”美味了。每当她把做好的果子晾晒在外面时,毛泽东和小伙伴们总要想方设法去偷着吃。

精明的邹四阿婆为防止小孩们偷吃,就架起梯子,把果子晾晒在屋顶上。这一下毛泽东和小伙伴们可犯难了,搬动梯子会被发现,用竹竿挑拨,要弄出响声。毛泽东正想办法,忽然看见几只飞起的蚱蜢,两只大眼睛一亮,这不是“天兵天将”吗?于是他们马上捉了几只大蚱蜢,用细长的绳子拴着蚱蜢锯齿般的长腿,顺风把它抛到房子上的果子里。这办法果然有效,他们牵动绳子,待蚱蜢正要起飞,突然往下一拉,蚱蜢那锯子般的腿就把一些果子弹了下来。小伙伴们忍住笑,抢着捡果子吃。

傍晚,邹四阿婆收果子时,发现稀少了许多。她心想,既没有起大风,鸡又飞不上去,地下也不见痕迹,这是怎么了?越想越糊涂:“想必是给背时的喜鹊、乌鸦吃了吧?”

毛泽东在私塾里也是如此。那是在1903年夏季的一个上午,塾师邹春培因有事需要外出,他就让塾生们在课堂上温习功课。毛泽东待大家读完了书,就提议去池塘游泳。这个池塘就在门口的不远处,碧波粼粼,是他们经常游泳玩耍的地方。

邹春培回来看见他的几个塾生赤身裸体在游泳,以为很不雅,便责备他们不该玩水。毛泽东很不服气,他拿出《论语》,引用上面的话说:

“‘暮春者,春服既成,冠者五六人,童子六七人,浴乎沂,风乎舞雩,咏而归。’夫子喟然叹曰;‘吾与点也。’您看看,孔夫子也是赞成到河里去洗澡的。”

邹先生听了,知道《论语》上确实有这么一段话,但一时又感到自己下不了台,就想了一个办法,叫塾生们对对子,说是对不出来就惩罚他们。他马上出了一个上联:

“濯足;”

毛泽东见大家你看看我,我看看你,都对不出来,便对了一个下联:

“修身。”

邹春培又出了一个上联:

“牛皮菜;”

毛泽东对了一个下联:

“马齿苋。”

邹春培见难不倒毛泽东,便要在背书上找茬惩戒他。他叫着毛泽东的名字,要他到前边去背书。毛泽东却坐在自己的座位上纹丝不动。按照私塾里的礼节,学生在课堂上背书要先站起来,走到先生的讲桌前站好,面向一旁,不能正视先生,然后开始背书。毛泽东对这一套繁文缛节早就不耐烦了。邹先生见他一动不动,气得脸色煞白,命令他到前边去,按老规矩背书。毛泽东只好搬着自己的凳子走到先生跟前,坐在凳子上以挑战的目光望着先生,对着快被他气晕的邹先生说:

“既然我坐着背书你也听得清楚,那为什么非要我站起来背呢?”

怒不可遏的邹先生用力拉他,要他站起来,毛泽东用力挣脱了。邹先生便气呼呼地来到上屋场向毛顺生告状,他说:

“你家咏芝了不得啦,他的才学比我高,我教不了啦!”

毛家的家教历来很严。毛顺生听说儿子带头闹学,和先生顶嘴,格外恼火。他顺手在路边捡了一根楠竹丫子,立刻跑到南岸私塾,见到儿子不由分说就劈头盖脸地打下来。毛泽东知道父亲的性子,脾气一上来,准是一顿死打,便急忙躲避,夺路而逃,一溜烟似的跑了。父亲怎么也追不上他,急得跺着脚直骂:

“我看你跑到哪里去!你敢回来,我就打死你这个没王法的东西!”

毛泽东心里明白,这时候回家,肯定还要挨一顿打,恐怕连母亲也劝不住,所以他决定暂时不回家,想跑到县城里去。此时的毛泽东还不到10岁,他在山里走了许久,也只是围着韶山打转,转了3天还没有走出韶山冲,离家也不过4公里左右,可就是不敢回家。父亲托人四处寻找儿子,最后还是毛氏家族一个砍柴老人在山里发现了他,毛泽东这才多少有点不情愿地回了家里。

后来毛泽东回忆起这件事,他说:

“我回到家里以后,想不到情形有点改善。我父亲比以前稍微体谅一些了,老师的态度也比较温和一些。我抗议的效果,给了我深刻的印象:这次‘罢课’胜利了。”

毛泽东一向反对父亲的刻薄、自私和专横,他和父亲一有争端,总要公开辩驳一番。有一次,毛顺生请了一些生意场上的人到家里作客谈生意,文七妹很快就准备好了酒菜,殷勤招待着客人们。毛顺生想让客人见见自己的长子,同时也想让儿子接触一下生意场面,结识生意中人,以后接自己的班,便大声喊道:

“石三,快来给客人斟酒!”

毛泽东正在看书,坐在那里一动不动,嘴里不耐烦地嘟囔着:

“我讨厌那些财迷心窍的买卖人,要斟,你自己斟!”

毛顺生见儿子不听话,觉得很没面子,顿时火了,放下筷子就要去打毛泽东。文七妹急忙过来拦着,劝道:

“石三已经长大了,做事情有他自己的主意,不要让他去做他不愿意做的事情。”

说罢,便笑容满面地给客人们一一斟了酒。毛顺生却依然怒气不息,当着满屋子客人的面骂儿子懒而无用。毛泽东被激怒了,他说:

“年纪大的人应该比年纪小的多做事,父亲的年龄比我大两倍多,所以应该做更多的事。等我到了你那年纪,会比你干的还要多。”

父亲又骂他不孝。毛泽东已经学会了引经据典,他说:

“经书上讲‘父慈子孝’,只有‘父慈’才能‘子孝’。为父不慈,哪来子孝?”

说完赌气跑出家门。母亲追出来劝他回去,父亲和客人们也赶来了。父亲一边骂一边责令他回去。毛泽东跑到一个池塘边,恫吓父亲说:

“如果你再走近一步,我就跳下去。”

毛顺生坚持要儿子磕头认错;毛泽东则表示,如果父亲答应不再打他,他可以跪一条腿磕头,因为另一条腿是属于母亲的。毛顺生见再僵持下去也不好收场,只好许诺说不再打他了,于是毛泽东就向父亲道了歉,磕了半个头。就这样,一场风波总算平息了。

毛泽东从这件事认识到,在压力面前决不能示弱,“唯有反抗才有出路”。

还有一件事,毛泽东再一次表现出了他那特有的倔强和睿智。有一天他在去唐家圫外婆家的路上,在走到石砚冲的山边时,突然闪出一个人来,双手叉腰挡住了去路。此人是一个姓赵的富家子弟,他平日里总喜欢欺侮贫弱,经常在穷人面前卖弄文墨,附庸风雅。他听说毛泽东聪明机智,便要借故难为难为毛泽东。他说:

“我知道你是文家的外甥,今天我要考考你,你若能回答出我提出的问题,就让你过去,若回答不出来,可别怪我赵某不客气,你就得大方框关小方框。”

“什么意思?”

毛泽东质问他。赵某神气地说:

“你若回答不出来,就得从我胯下爬过去。”

毛泽东一听,不慌不忙也来了一个双手叉腰,说道:

“你爱问就问吧!”

赵某说:

“百家姓里的‘赵钱孙李’,分开怎么解释,合起来是什么意思?”

毛泽东稍加思索,便说:

“‘赵钱孙李’分开是:赵公元帅的赵,有钱无钱的钱,龟孙的孙,有理无理与李同音。大宋天子赵匡胤说过:有钱龟孙不讲理!”

他不仅按照要求回答了赵某的刁难,而且还巧妙地骂这个富家子弟是“有钱龟孙不讲理”。赵某人一听,禁不住满脸怒气,可他又挑不出毛病来,也不好发作,只得让毛泽东过去了。

这正是:韶峰石头多棱锷,为柱长天不折琢!

欲知毛泽东后来如何发展,请看下一章便知。

东方翁曰:反抗压迫,追求平等,不唯少年毛泽东,人人如此,天性使然也。童年的毛泽东也像其他的孩子一样,不过是一个“顽童”而已。可一般孩童在经过长期的社会“历练”之后,基本上都已经被琢磨得没有棱角了,圆滑了。而毛泽东的不同于常人之处就在于,他以他那叛逆的个性,对父亲的打骂实行反抗;对老师的指责据理力争;对邪恶势力的压迫进行反击;在强大的陈旧的习惯势力和影响面前,始终没有被驯服,没有被琢磨得像大多数人一样,反而把他那种可贵的理性的倔犟性格,即天生的斗争思想萌芽,随着自身素质的不断提高,一步一步地升华为马克思主义的斗争哲学——与天奋斗,其乐无穷;与地奋斗,其乐无穷;与人奋斗,其乐无穷。更为难能可贵的是,他还始终将这种斗争哲学作为自己的座右铭,终生实践,至死不悔。

\end{document}
