\documentclass[../../dazhuan.tex]{subfiles}
% 第一卷
\begin{document}
\chapter*{第二章}
\pdfbookmark{第二章}{V01C02}

\begin{pref}
	在我们的心目中,彭铁匠是第一个农民英雄。
\end{pref}

话说在1904年秋,毛泽东转学到了关公桥私塾,塾师叫毛咏生。

毛泽东在这个私塾里读书的时候,有一个名叫黑皮伢子的小同学,因家里很穷,常常不带中午饭,只好饿着肚子,一直到下午放学了,回到家里才再吃上一餐晚饭。毛泽东很同情这位同学,就常常把自己带的午饭分一半给他吃。

文七妹看到儿子晚饭老是吃得那么多,大惑不解。毛泽东就告诉母亲说,她精心准备的午饭是和同学分着吃了。从此,善良的母亲每天早晨都要让儿子带上两份午饭去上学。

1904年这一年,毛顺生手头的钱积攒多了,就买进了家境十分贫苦的堂弟毛菊生的7亩水田。这样一来,他的田产就增加到了22亩,每年可以收稻谷80担。

毛泽东和母亲同情毛菊生的苦楚,都反对这样做。毛顺生则认为,这没什么不应该的。他说:

“管他兄弟不兄弟,我这是用钱买来的田。”

这件事留给毛泽东的印象太深了,以至多年以后他还念念不忘这件事情。

1905年春,毛泽东又先后转入桥头湾和钟家湾私塾就读。钟家湾的塾师叫周少希。

这一年,毛泽东害了一场大病,在母亲的多方护理调养下,才慢慢地好转了,只是他的身体更加瘦弱了。毛泽东由此想到,如果身体长久不好,将来什么事也不能做,那是多么可怕的事情呀!于是他下了决心,一定要把身体锻炼好。

1906年秋,毛泽东转到5里外的井里湾私塾就读,塾师叫毛宇居。

毛宇居生于清朝光绪七(1881)年,与毛泽东是同一个太祖。他谱名毛泽启,字宇居,又叫禹居,别号韶麓散人。其父毛福生系前清国子监生。毛宇居博学强记,还从过军,曾经跟随蔡锷的部队转战到云南、四川等地。此人见多识广,在韶山冲也算得上是凤毛麟角的人物,乡亲们都称他是“纯儒”;又因他擅长诗文和书法,乡亲们还送给他一个“韶山一支笔”的雅号。

毛泽东在井里湾私塾开始跟着毛宇居练习书法,从学唐初的欧阳询入手,后来改学书体刚劲开放的清代书法家钱南园。在主要学业方面依然是读《四书五经》,可他不喜欢这些书。他喜欢读的是《公羊春秋》、《左传》等经史书籍,这使他对中国历史产生了浓厚的兴趣。他更喜欢读的是中国古典小说,特别是关于造反的故事。他曾背着老师读过《精忠传》、《水浒传》、《三国演义》、《西游记》、《隋唐演义》等。后来毛泽东说自己喜欢读中国古典小说,认为这些书对自己影响很大。

毛宇居回忆说:“当时私塾的规矩,认为小说是杂书,不准学生看。因此,他总是偷着看,看见我来了,就把正书压在上面。后来被我发觉了,就故意多点书,叫他背,但他都背得出来。”

思想活跃的毛泽东不愿受封闭式的教育方式的束缚,尤其是对呆板的死记硬背的学习内容,深感不满。

有一次,毛宇居先生有事外出,行前再三叮嘱塾生们要老老实实地在教室里读书,不许私自走出私塾。在毛宇居先生走后,毛泽东就像逃出笼子的小鸟一样,跑到屋后的山上,边背书边摘毛栗子,书背熟了,毛栗子也摘了一包。回到课堂上,他给每个同学送上几颗,自然也孝敬了先生一份。毛宇居却不领情,责怪他说:

“你怎么敢私自跑出去玩呢?”

毛泽东回答说:

“闷在屋里,头昏脑胀,死背硬读也是空的。”

“放肆!”

他见先生真的生气了,只好说:

“那你叫我背书好了。”

毛宇居知道背书难不住他,便指着院子里的天井,气冲冲地说:

“这回既不打你板子,也不罚你背书,你给我赞一赞这天井!”

毛泽东将天井观察了一下,略加思索,脱口吟道:

“天井四四方,周围是高墙。清清见卵石,小鱼囿中央。只喝井里水,永远养不长。”

毛宇居听了很是赞赏,也就不再惩罚他了。

1907年,毛泽东已经14岁了,他的四弟毛泽民11岁,六弟毛泽覃才只有2岁;家里还雇佣有长工和短工。要给这么多的人做饭并操持其它家务,年已40的文七妹,其劳苦程度是可想而知的。一生精明的毛顺生,此时与其说是需要一个得力的帮手,不如说是急需一个能够操持家务的好的劳动力。于是他便按照湖南“亲上加亲”的传统习惯,给毛泽东定下了一桩娃娃亲,女方是韶山杨林炉门前罗鹤楼的长女罗氏。罗氏名叫罗一秀,生于1889年10月20日,比毛泽东大4岁两个月零6天。

毛、罗两家原是世交,上两辈人就有亲戚关系。罗一秀的祖母毛氏,是毛泽东的祖父毛翼臣的堂妹。更重要的是罗家有田产,比较富裕,也不乏读书之人,在当地有较高的声望。在毛顺生看来,毛、罗两家做亲可谓是门当户对了。

未成年的毛泽东自然是反对这门亲事的,但毕竟胳膊拧不过大腿,他为了顾及父母和亲友的面子,只好默默地忍受着这桩“痛苦的婚姻”。在成亲那一天,他这个衣着整齐的小新郎,忍受着那种僵化而可恶的仪式,向每一位来宾叩拜。

年方18岁的罗一秀,性情温顺善良,生得脸庞圆润,体态丰满,是一位能操持家务的好手。毛泽东的父母看在眼里自然是十分满意的。但毛泽东始终不承认罗一秀是自己的妻子,坚决拒绝与这位比他大4岁的新娘圆房,也从未碰过她一根手指头,弄得父亲十分恼火,却也无可奈何。

毛泽东在1936年曾经对斯诺说:“我14岁的时候,父母给我娶了一个20岁(中国传统计岁法都是按虚岁计算——笔者注)的女子,可是我从来没有和她一起生活过——而且后来也一直没有。我不认为她是我的妻子,当时也几乎没有想到过她。”

后来这位罗氏女子,不幸于1910年2月11日因患细菌性痢疾,不治身亡,年仅21岁。她被葬在韶山冲南岸的土地冲。几十年后,湘潭毛氏族人依然按照旧传统、旧习惯,把罗一秀作为毛泽东的原配夫人,载入《毛氏族谱》。更为荒唐的是,族人们因罗一秀无嗣,便将毛泽东与杨开慧所生的已经不在人世的第3个儿子毛岸龙,作为罗一秀的儿子载入族谱。《毛氏族谱》上写道:“承夫继配杨氏子嗣”,“远智,与原配罗氏为嗣”。

且说毛顺生在持家方面克勤克俭,他带着一家人勤奋劳作,家里基本上没有吃闲饭的人。儿子年纪小的要干一些割猪草之类的轻活,大一点的就要到田地里去干活。加上他经营有方,所以他的家业日益发达,到了1907年,他家的土地就增加到24亩。在攒了一笔钱之后,毛顺生就不再买进土地了,而是典当别人的田地雇人耕种,因为典地要比买地便宜,自己还不用操心农事,便可以取利。

毛家每年大约要消费掉4500斤大米,此外还有约7000斤余粮可以出售,精明的毛顺生便开始做起了贩卖粮食、耕牛和生猪的生意,赚了不少的钱。他还放出了高利贷。在运输上,他也逐步由肩挑改为用车运送,后来更发展到从银田寺雇船,将米运往湘潭市易俗河一带去出售。一开始,这种生意还是小规模的,后来毛顺生得到岳家亲戚的贷款帮助,便到湘乡大坪坳一带成批购进稻谷,加工销售。毛家已经变得像模像样了,有一座牛棚,一个粮仓,一个猪圈,还有一个小小的磨坊。

毛顺生也算得是一个财东了,他就在银田寺的“长庆和”米店入了股,并同“祥顺和”、“彭厚锡堂”等店铺有商务往来。为了流通方便,自家还印制了“毛义顺堂”的票子,同“吉春堂”的纸票流通周转。“吉春堂”是湘乡大坪坳一家设有药材、肉食、杂货等几个店铺的大商号,老板姓赵,是毛顺生妻兄文玉瑞的岳家。就为了这层生意关系,后来毛顺生一手包办了小儿子毛泽覃与赵家女儿赵先桂“亲上加亲”的婚姻。

1936年毛泽东同斯诺谈到了他家的经济情况,说相当于“富农”的地位。1950年隆冬,韶山乡土地改革划成份时,毛泽东为自己家划的成份是“富农”。

且说文七妹与毛顺生一心发家致富出人头地的做法相反,她则是为毛泽东弟兄的长大成人费尽了一片苦心。她对孩子们的教育,常常寓于循循诱导之中。文七妹性情温和,心地善良,富有同情心,乐于助人。每逢荒年旱月,文七妹就悄悄送一些米粮接济贫苦的乡亲们,这和过于自私的毛顺生有着很大的反差。对毛泽东弟兄3个影响最大的就是她这种高尚的品德。

毛泽东在延安与斯诺谈话时说:“我母亲是个心地善良的妇女,为人慷慨厚道,随时愿意接济别人。她可怜穷人,每逢灾荒,一些穷人前来讨饭的时候,她常常给他们饭吃。但是,如果我父亲在场,她就不能这样做了。我父亲是不赞成这样做的。我家为了这事多次发生过争吵。”

“我家分成两‘党’。一党是我父亲,是执政党。反对党由我、母亲、弟弟组成,有时连雇工也包括在内。可是在反对党的‘统一战线’内部,存在着意见分歧。我母亲主张间接打击政策。凡是明显的感情流露或者公开反对执政党的企图,她都批评,说这不是中国人的做法。”

在母亲的影响下,少年时代的毛泽东对穷苦人也非常同情,养成了乐于助人的品性。在韶山至今还流传着他许多同情和帮助穷人的故事。

那是在8月的一个中午,毛泽东顶着火辣辣的太阳去外婆家,行至湘乡五里牌,看见一位挑着一担盐的老汉艰难地向前移动着步子,他就赶上去问:

“老伯,你挑这么多的盐到哪里去?”

“我到凤音四都!”

毛泽东从袋子里拿出一条裤子,说:

“我也去凤音四都,帮你背一点吧!”

老汉看着眼前的小伙子很真诚,就说:

“那太好了。”

毛泽东便把裤脚扎好,满满地装上盐,扛着绕道走了十多里,跟着老汉送到凤音四都石坝咀,主人家要请他吃饭,他说啥也不吃就走了,回到唐家圫时已经是晚上掌灯时分了。

毛顺生的堂弟毛菊生家境十分贫困,毛泽东就常常和母亲一起接济他们。每年到了最难过的年关时节,他和母亲都会背着毛顺生给毛菊生一家送去白米和腊肉。

有一次,一个农民秘密会党哥老会的一些成员来到毛家偷走了不少东西。毛泽东说:

“这是件好事,因为他们偷到了他们没有的东西。”

这种大逆不道的观点,不仅遭到了父亲的反对,连他的母亲也不赞成。

在一个大雪纷飞寒风刺骨的冬天,毛泽东在去私塾的路上遇见了一个衣服单薄的少年,他见那位少年在风雪中冻得浑身直打哆嗦,就询问他的家境状况,当了解到这位少年家里十分贫苦后,非常同情他,就脱下了自己身上的一件衣服让他穿上。

毛泽东最讨厌的事情,是去为越来越富的父亲四处讨账。在一个旧历年关时节,父亲又要他去收一笔卖猪的账,毛泽东违拗不过,只得去收了。在他回家的路上,碰到了一群衣衫褴褛的穷苦人,顿生怜悯之心,于是就把刚收来的钱全部分给了那些人,回到家里,自然是挨了父亲一顿臭骂。

有一回,毛泽东受父亲之托去买一个老阿婆养的猪,讲好价钱后先付了一元定金。没过几天,父亲又要他去把那头猪赶回来。此时的猪价又涨了,老阿婆后悔自己卖早了,受了损失。毛泽东很同情她,就没有再坚持要那头猪,只收了老阿婆退还的一元定金。父亲见他没有把猪赶回来,责问他是怎么回事,毛泽东只好如实讲了,父亲非常生气地骂道:

“你呀!有钱不赚是傻子,是败家子!”

一天下午,毛泽东眼看着一场暴风雨即将来临,急忙去抢收自家场上的谷子。他突然看见在邻居的晒谷场上,只有毛四阿婆一个人在那里手忙脚乱地忙活着,就立刻赶上去帮忙,结果自家的谷子却被雨水冲到水沟里去了。父亲见他浑身水淋淋地跑回来了,非常生气,骂道:

“自己的谷子流到沟里去了,你倒好,胳膊肘子往外拐,帮人家去收。”

骂完了还不算,他扬手就要打。毛泽东站在那里一动不动,理直气壮地说:

“人家家里很苦,是佃的田,还要交租,冲去一点都了不得。我们是自己的田,又比人家多些,冲走一些自然也不大要紧的。”

父亲听了更加冒火,说:

“你说不要紧,你还吃饭不吃饭?”

毛泽东笑着说:

“好喽!我一餐少吃一口,这总可以了吧!”

1907年这一年,毛泽东已经读完了5年的私塾,在他和父亲发生了一次冲突后,就不再上学了,待在家里务农,一直到1908年。他白天同成年人一样在田间劳动,跟着长工毛春成学会了扶犁、掌耙、扬谷、下种等农活,晚上则成了父亲的管账先生。

毛泽东曾回忆说:“我刚识了几个字,我父亲就让我开始记家里的账。他要我学珠算。由于我父亲对这事很坚持,我就开始在晚间记账。他是一个严厉的监工,看不得我闲着,如果没有账可记,他就叫我去干农活。他是一个脾气暴躁的人,常常打我和我的弟弟。他一文钱也不给我们,而且给我们吃最次的饭菜。他对雇工们做了让步,每月逢十五在他们吃饭时给鸡蛋吃,可是从来不给肉吃。对于我,他既不给肉吃也不给鸡蛋吃。他的严厉态度大概对我也有好处,这使我干活非常勤快,使我仔细记账,免得他有把柄来批评我。”

当然,毛泽东的兴趣并不在这吃的上面。他晚上记完了账,就躲进自己的卧室里点起桐油灯,读起书来。父母住在隔壁,总是催他早点睡觉,第二天早上好去做工。毛泽东听到催促声,便用布遮住灯光。在中国农村绝对是应该睡觉的时候了,可毛泽东却依然俯身坐在那里,面前放着的不是描写绿林好汉的小说《水浒传》,就是吴承恩的《西游记》。他那汗流满面的脸,贴近那盏火苗只有黄豆大小的油灯,并且还要用被子半遮着油灯和自己,以免灯光透出去。因为父亲是不喜欢他夜里点灯浪费桐油的。毛泽东一看就是三更半夜,父亲总是喊他:

“快早点睡!明天早起还要做工夫!”

母亲有时候也催他:

“快点睡吧,莫熬夜了,这样会把身体搞坏的。”

毛泽东也总是回答说:

“好,就睡了,就睡了。”

父亲很快就看穿了他的把戏,起床后看到儿子还在读书,就吼道:

“哪里这样的喊不赢哪!你一夜熬掉了多少桐油,一个月下来就是几百文铜钱,这样下去那还了得?”

父亲说着把书翻过来一看,竟是一本《水浒传》,就更生气了。

有一天上午,毛顺生看到儿子在地边的一块墓碑旁看小说,就大发雷霆,说:

“你是不打算干活了?”

说着扫了一眼儿子身边的两只空粪筐,又责备他一上午也没有从猪圈里往田里送一筐粪。毛泽东说:

“爹,我不是不干,送了五六筐了,我只是歇一小会儿。”

到了傍晚,毛顺生发现儿子又在那块墓碑旁看那些离经叛道的东西,就责备他被那些坏书教坏了,以至于连父亲的警告都不屑一顾了。毛泽东说:

“爹,我是听你的话的,你要我做的事我都照做了。活我要照常干,书也要照常读。”

毛顺生看见儿子一下午确实送了15担粪肥,本来不高兴的他也吃惊地张大了嘴巴。

毛泽东对父亲时时顶撞,可对母亲却是非常孝顺的。有一次母亲因生病许下了一个大愿,他便专程替母亲去南岳朝拜上香,一路上手里拿着个小凳子,走十来步就要跪拜一次,嘴里还要喃喃念叨着:“南岳圣帝,阿弥陀佛。”就这样竟走了20多公里的路程。

后来毛泽东所读的书籍越来越多了,母亲对他的影响就不如以前那样大了,他对佛教的信仰日趋淡化,这使母亲感到很不安。

且说毛泽东在家务农这一时期,如饥似渴地阅读他所能找到的一切书籍。有一天,他忽然想到了这样一个问题,中国的小说和故事里有一点很奇怪,里面没有描写种田的农民,所有的主要人物都是武将、文官、书生,从来没有一个农民做主人公。他对此困惑不解,问了许多人也都回答不出来。这个问题一直困扰了他两年。后来他仔细一分析,这才明白书中颂扬的那些人都是统治者,都拥有大量的土地,他们控制着土地让农民替他们耕种,自己是不用劳动的。

正是:\begin{xemph}德承萱堂怜孤贫,初羡草莽恨不平。

\hspace{3em}书中常见将和相,为何难觅种田人?\end{xemph}

毛泽东读完了在韶山所能借到的书,就跑到唐家圫外婆家从表兄文运昌那里借书读。表兄文运昌告诉他说:

“相公借书,老虎借猪。所以还是要先打条子后拿书啊!”

于是,毛泽东每次向文运昌借书都要打借条。他在还书时,还常常把卷了角的书放在凳子上,用手小心压平,或用屁股坐平,然后“完璧归赵”。

自从看了表哥借给他的书籍,毛泽东可是大开了眼界,那些关于列强瓜分中国的小册子等进步书籍和外面世界所发生的事情,使刚刚步入青年的毛泽东开始有了一些政治意识。改良主义者郑观应所著的《盛世危言》,是毛泽东非常喜欢的一本书。这本书的作者以为,中国之所以弱,在于缺乏西洋器械,如铁路、电话、电报、轮船等,所以他提出应当把这些东西输入中国。郑观应还在书中提出了“主以中学,辅以西学”的主张。

毛泽东又读了另一位改良主义者冯桂芬著的《校邠庐抗议》。此书对外国侵略和清政府的腐败表示了不满,书中谈到了日本占领朝鲜、台湾的经过,谈到了越南、缅甸等地宗主权的丧失,并提出了一些富国强兵的主张。读了这本书后,毛泽东对祖国的前途感到沮丧。他开始认识到:“国之兴亡,匹夫有责。”几十年后,他还能充满激情地回忆起这本书开头的第一句话:“呜呼!中国其将亡矣!”

毛泽东的视野被打开了,这激起了他要恢复学业的愿望。但父亲对儿子读这类书和小说是不赞成的,他认为读这些书是浪费时间。

不久,毛顺生因为山林纠纷与人打了一场官司,本来对方是没理的,但人家知书识墨,在大堂上引经据典,竟然把无理说成了有理;而他自己则是有理说不清,一片山地就这样被人家占去了。由此他开始希望自己的儿子也能熟读经书,以后可以帮助他打官司。后来毛泽东继续求学能得到父亲的同意和支持,这是其中的一个重要原因。

1909年秋,毛泽东终于复学了,他就读于乌龟井私塾,拜毛简臣为师,攻读《春秋左传》。

毛简臣毕业于法政学堂,以善于撰写诉讼文书著称于韶山一带。

1910年春,毛泽东又到堂伯父毛麓钟在东茅塘自办的私塾里读书。

毛麓钟,谱名贻训,字麓钟,号云阁,生于1866年,是韶山冲唯一的秀才。他曾对弟子们说:“美雨欧风向我神州冲击,惟有迎头赶上,才能自立于世界之林。”这一类话对毛泽东具有深刻的影响。毛泽东在这里选读了《纲鉴类纂》、《史记》、《日知录》等古籍,也读了许多时论和一些新书。

在辛亥革命后,毛麓钟曾经在蔡锷部下供过职,转战于云南、四川两省,不久因疾回乡;1921年病逝于东茅塘。

且说在1910年春,韶山冲清溪李家屋场从外地回来一位维新派教师李漱清,他常常给韶山冲里的人们讲述一些在外地的见闻和维新故事。他还主张反对佛教,提出“弃庙兴学”,把祠堂和寺庙的田产拿出来兴办新式学堂。这种新式学堂不同于过去的那种私塾,不拜孔夫子,不读四书五经,而是要学一些新的科学知识。韶山冲里的人们对李漱清的言论议论纷纷,很少有人支持他。而毛泽东却赞成他的主张,经常到他家里去,拜他为师,向他求学问教,与他谈心,还向他借书看,以便了解外面的新鲜事情。他们成了好朋友。李漱清支持毛泽东多读书,热情地向毛泽东推荐新书。他还去劝说毛顺生,要让毛泽东多读些书。

就在这个时候,湖南长沙发生了一件重大事情,给毛泽东留下了深刻的印象。

有一天,毛泽东和他的同学碰到了一群从长沙来的贩卖开花蚕豆的湘乡籍商人,那些商人说他们之所以离开长沙,是因为长沙发生了大规模的抢米暴动。  

原来在1910年4月份,湖南省正闹着粮荒,长沙的饥民们成群结队地来到巡抚衙门前请愿,要求官府救济,平价粜米,可是他们得到的却是巡抚岑春蓂的无理答复。饥民们被激怒了,他们冲进衙门,砍断旗杆,赶走了岑春蓂。后来清政府又派来一名巡抚,野蛮地镇压了饥民暴动。许多饥民被斩首示众,他们的首级都被挂在旗杆上,以儆效尤。

笔者在这里需要说明的是,在有关这一事件的记载中,湖南巡抚的名字多有讹误,本传在第一版中也受到了影响。据新疆叶才林考证:《毛泽东的故土情》一书的作者胡长明博士误把湖南巡抚岑春蓂写作了岑春萱,一来可能是因“蓂”与“萱”这两个字字形笔画相近,二来可能是受了“岑春煊”这个名字的影响。《中国历史大事编年》一书在“长沙抢米风潮”条目中则直接将“岑春蓂”写成了“岑春煊”。岑春煊曾任两广总督,岑春蓂曾任湖南巡抚,二人虽然同为云贵总督岑毓英之子,却是不能“弟冠兄戴”的。

且说毛泽东和同学们在私塾里谈论着这一来自山外的惊雷般的消息,他们一直议论了好多天。毛泽东还把这件事同韶山人民联系起来,他觉得造反的人和家乡的人们一样,都是穷苦的老百姓,他们抢米是官府逼出来的,因此他对被杀害的饥民们非常同情。

在此之前,毛泽东就听说过韶山也发生过穷人造反的故事,那是哥老会的成员因为地租问题与一个地主发生了纠纷,地主用银元贿赂官府赢了官司。哥老会成员们忍无可忍,便在一位姓彭的铁匠领导下举行了暴动。那个地主造谣说,彭铁匠他们在起义前杀了一个婴儿祭旗。湖南巡抚很快便派来官兵弹压,彭铁匠和哥老会成员们被击溃了,他们只好躲进了附近的浏山,但不久还是被围捕了,彭铁匠也被官府斩首示众。在毛泽东看来,这是《水浒传》里的故事在他家乡的又一次重演。他后来回忆说:

“在我们的心目中,彭铁匠是第一个农民英雄。”

再说长沙抢米风潮很快也波及到了发生粮荒的韶山地区。不久,毛泽东的家也成了造反的对象。韶山的饥民们一双双饥饿的眼睛都盯着商人和富人们的粮仓,他们喊出了“吃大户”的口号。就在这种情况下,毛顺生居然还往长沙贩卖粮食,愤怒的饥民们拦截了他的货船,把粮食抢个了精光。只气得毛顺生暴跳如雷,可毛泽东并不同情他父亲。他在26年后追忆这件事时还说:“我学会了恨他。”

毛泽东认为,越来越富的令人厌恶的父亲,是旧中国不平等社会秩序在当地的捍卫者。他得出了一个可怕的、骇人听闻的结论:“老头儿是中国自救之路上的一只拦路虎。”

在韶山这次闹粮荒中,毛泽东更是站到了群众斗争的第一线。

这也是在1910年农历三月间,族长毛鸿宾把族上100多担谷子私下高价卖给米商,以获取暴利。族人发现后纷纷找到毛鸿宾,要求开仓平粜。毛鸿宾倚仗族长的权势,把领头的佃农毛承文捆了起来,扬言要在祠堂里按照族规惩罚他。

毛泽东闻讯赶到祠堂,和毛鸿宾当面辩论。他慷慨陈词,说违反族规的不是要求平粜的毛承文,而是私下高价卖谷的族长本人。愤怒的群众纷纷站到毛泽东一边,呐喊助威。毛鸿宾自知众怒难犯,只得放了毛承文,并答应交还私自卖出的稻谷。

毛顺生因儿子与族长公开叫板,闹了毛家祠堂,害怕毛鸿宾报复,便决定让儿子停止读书,去湘潭一家同他有来往的米店当学徒。他来到东茅塘私塾,向先生说明了原委,又对毛泽东说:

“你莫读书了,快点把东西收拾好,马上跟我回家去!”

毛泽东对父亲的决定很不理解,闷闷不乐地跟随父亲离开了私塾。父亲在路上对他说:

“你呀,就莫给我再闯祸了,快点走出韶山冲,明天就送你到湘潭去,我和老朋友讲好了,让你到宽裕枯粮行去当个徒弟,学点成家发财的真本领!”

毛顺生所说的这家“宽裕枯粮行”的老板叫毛槐林,是他的堂兄,两人关系甚为密切。毛槐林在湘潭经营的宽裕枯粮行位于通往窑湾的江畔,是一座单开间四进砖壁木板3层楼房,宽仅两米多,长约20米左右,外墙很厚,房间中规中矩。这个枯粮行不是米店,不直接销售加工好的米粮,只经营未加工的花粮、枯饼和黄豆等的贸易批发。

毛泽东听父亲说要他到湘潭去,心里又惊又喜。他还没有离开过家乡,早听人说,湘潭有繁多的水陆码头,有一条十几里的长街紧靠着湘江;江里很多洋船,就像两层楼的房子那样高。船是铁做的,可在水里也不会沉。他曾跟毛麓钟先生读过一些唐诗宋词,不少是赞美湘江的:“欸乃一声山水绿”、“为谁流下潇湘去”,这些诗句,给他留下了很深刻的印象。他真想马上去看美丽的湘江,看看湘潭这个繁华的口岸。于是,他就到毛槐林的宽裕枯粮行干了几个月。

后来到了1936年,毛泽东在与斯诺谈话时说到了这件事,他说:“我父亲决定送我到湘潭县一家同他有来往的粮店里去当学徒。起先我并不反对,觉得这也许是有意思的事。可是差不多就在这个时候,我听说有一个与众不同的新式学校,于是决心不顾父亲的反对,到那里去上学。这个学校就在我母亲娘家所在的湘乡县。”

欲知毛泽东能否到湘乡县的新式学校学习,且看下一章叙述。

东方翁曰:毛泽东自幼生在农村,长在农村,目睹了农民的艰辛劳作。正是由于那些脚踩牛粪的劳动者艰苦卓绝的奉献,才为社会创造了巨大的财富,极大地丰富了人类的生活。而在毛泽东喜欢看的那些书中,却几乎看不到农民的影子,满眼全是吃着农民、喝着农民、享受着农民劳动成果的帝王将相和才子佳人。于是他愤愤不平了,于是他揭竿而起了;于是就有了满怀激情的《在延安文艺座谈会上的讲话》,有了文化部是“被死人统治着”的牢骚,有了“为人民服务”的座右铭。黄巢诗曰:“他年我若为青帝,报与桃花一处开”。他能不能言行如一,已经无从知道了。但众所周知的是,毛泽东的确是做到了。翻遍中华五千年历史,可有第二个出身于草莽的胜利者对农民尚有如此深厚的情愫吗?某孤陋寡闻,至今未见!



\end{document}
