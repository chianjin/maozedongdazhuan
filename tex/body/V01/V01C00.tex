\documentclass[../../dazhuan.tex]{subfiles}
% 第一卷
\begin{document}
\chapter*{引\quad 子}
\pdfbookmark{引子}{pref}

词曰:

\begin{xemph} 
独立寒秋,湘江北去,橘子洲头。看万山红遍,层林尽染;漫江碧透,百舸争流。
鹰击长空,鱼翔浅底,万类霜天竞自由。怅寥廓,问苍茫大地,谁主沉浮?

携来百侣曾游,忆往昔峥嵘岁月稠。恰同学少年,风华正茂;书生意气,挥斥方遒。
指点江山,激扬文字,粪土当年万户侯。曾记否,到中流击水,浪遏飞舟?
\end{xemph}

这一首世人公认的著名词章:《沁园春・长沙》,是出自世界上一位最伟大的历史人
物之手。作者在20世纪20年代中期填这首词时,还是一位年轻的职业革命者。那时候
的他,独自站在橘子洲的一端,望着滔滔的江水,日夜不息地向北流去,湘江上那美
丽动人的自然秋景,并不能使他忘怀他所面对的严峻的革命形势。

他所处的那个时代,一方面是工农革命运动蓬勃发展,一方面是反动势力为了维护他
们的反革命统治对革命力量进行疯狂的反扑。忧国忧民的年轻革命者,自然是思绪激
荡,心潮澎湃,禁不住发出了一个惊天之语:

{\centering\emph{问苍茫大地,谁主沉浮?}\par}

这个问题,当然无需他人作答,在这位年轻人的心中,早已有了一种非常强烈的呼唤:
我们!只有我们!才是未来世界的主宰者!

此时此地的他,自然而然想到了那些年轻的伙伴们。曾几何时,大家一起指点江山,
激扬文字,粪土当年万户侯!如今,更需要同志们再度携手,勇往直前,在革命的大
潮中奋勇搏击,砥柱中流!  
                               
是啊,当他还在湖南一师读书的时候,就与学友们一起创立了新民学会,制定了“革
新学术,砥砺品行,改良人心风俗为宗旨”的人生初级目标。当他走出学校大门步入
社会的时候,就投身到汹涌澎湃的革命浪潮之中,与孙中山、李大钊、陈独秀等人
共商国是;接着,他又在国民党中央代理宣传部长的位置上,与那些新老右派们唇
枪舌剑地斗了一番。

后来这位年轻的革命者一生奋斗不息,果然成就了一番亘古未有的大事业。

他戎马22年,打破了蒋介石的多次军事围剿,驱逐了日本帝国主义出中国,倾覆了
蒋家王朝的独裁统治,统一了华夏大地。  

他治国27年,在中华大地上开天辟地第一遭废除了几千年的私有制,建立起一座由
人民当家作主的公有制的理想的政治大厦;并且巧妙而成功地改变了世界的政治格
局,使中华人民共和国巍然屹立于世界民族之林。
 
他一生料事如神。在他的青年时代直至暮年,数十年间,作为一个伟大的预言家,
凡大事件曾经预言多多,结果,几乎无一出乎其意料之外。

他一生反对英雄创造历史的观点,所以我们无须把他看作是英雄。可是他凭什么
能够做出这么一番惊天动地的大事业呢?说白了,除了他个人具有良好的素质和
崇高的革命理想之外,还有一个法宝,那就是一生广交学友、朋友、同志,发动
和依靠人民战争。

尽管在他身后对于他一生的功过是非数十年来人们褒贬不一,但他一生的行事为
人,应该说是无愧于中华民族伟大赤子这一光荣称号的。特别是在二十一世纪的
今天,读一读他的故事,无论是对于我们现在,还是将来,无疑都具有深远的历
史意义。 

读者诸公若心存疑虑,请稍安勿躁,待您耐心读完大传,自然就心领神会而决非
是笔者妄言了。剥曹雪芹诗一首,赠读者:

\emph{满纸真人事,一把心酸泪。莫道笔者痴,请解其中味。}
   
\end{document}