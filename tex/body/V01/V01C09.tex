\documentclass[../../dazhuan.tex]{subfiles}
% 第一卷
\begin{document}
\chapter*{第九章}
\pdfbookmark{第九章}{V01C09}
\begin{pref}
	以欂栌之材,欲为栋梁之任,其胸中茫然无有,徒欲学古代奸	
雄意气之为,以手腕智计为牢笼一世之具,此如秋潦无源,浮萍
无根,如何能久?
\end{pref}

话说在1917年8月23日,毛泽东给北京的黎锦熙老师写了一封信,他写道:
\begin{xquote}
\noindent 邵西先生阁下: 

省城一面,几回欲通音问,懒惰未果。近日以来,颇多杂思,四无亲人,莫可与语。弟自得阁下,如婴儿之得慈母。盖举世昏昏,皆是斫我心灵,丧我志气,无一可与商量学问,言天下家国之大计,成全道德,适当于立身处世之道。自恸幼年失学,而又日愁父师。人谁不思上进?当其求涂不得歧路彷徨,其苦有不可胜言者,盖人当幼少全苦境也。今年暑假回家一省,来城略住,漫游宁乡、安化、益阳、沅江诸县,稍为变动空气,锻炼筋骨。昨16日回省,20日入校,22日开学,明日开讲。乘暇作此信,将胸中所见,陈求指答,幸垂察焉。 

今之天下纷纷,就一面言,本为变革应有事情;就他而言,今之纷纷,毋亦诸人本身本领之不足,无术以救天下之难,徒以肤末之见治其偏而不足者,猥曰吾有以治天下之全邪!此无他,无内省之明,无外观之识而已矣。己之本领何在,此应自知也。以欂栌之材,欲为栋梁之任,其胸中茫然无有,徒欲学古代奸雄意气之为,以手腕智计为牢笼一世之具,此如秋潦无源,浮萍无根,如何能久? 

今之论人者,称袁世凯、孙文、康有为而叁。孙、袁吾不论,独康似略有本源矣。然细观之,其本源究不能指其实在何处,徒为华言炫听,并无一干竖立、枝叶扶疏之妙。愚意所谓本源者,倡学而已矣。惟学如基础,今人无学,故基础不厚,时惧倾圮。愚于近人,独服曾文正,观其收拾洪杨一役,完满无缺。使以今人易其位,其能如彼之完满乎?天下亦大矣,社会之组织极复杂,而又有数千年之历史,民智污塞,开通为难。欲动天下者,当动天下之心,而不徒在显见之迹。动其心者,当具有大本之源。今日变法,俱从枝节入手,如议会、宪法、总统、内阁、军事、实业、教育,一切皆枝节也。枝节亦不可少,惟此等枝节,必有本源。本源未得,则此等枝节为赘疣,为不贯气,为支离灭裂,幸则与本源略近,不幸则背道而驰。夫以与本源背道而驰者而以之为临民制治之具,几何不谬种流传,陷一世一国于败亡哉?而岂有毫末之富强幸福可言哉? 夫本源者,宇宙之真理。天下之生民,各为宇宙之一体,即宇宙之真理,各具于人人之心中,虽有偏全之不同,而总有几分之存在。今吾以大本大源为号召,天下之心其有不动者乎?天下之心皆动,天下之事有不能为者乎?天下之事可为,祖国有不富强幸福者乎?然今之天下则纷纷矣!推其原因,一在如前之所云,无内省之明;一则不知天下应以何道而后能动,乃无外观之识也。故愚以为,当今之世,宜有大气量人,从哲学、伦理学入手,改造哲学,改造伦理学,根本上变换全国之思想。此如大纛一张,万夫走集;雷电一震,阴曀皆开,则沛乎不可御矣!

近阅书报,将中外事态略为比较,觉吾国人积弊甚深,思想太旧,道德太坏。夫思想主人之心,道德范人之行,二者不洁,遍地皆污。盖二者之势力,无在不为所弥漫也。思想道德必真必实。吾国思想与道德,可以伪而不真、虚而不实之两言括之,五千年流传到今,种根甚深,结蒂甚固,非有大力不易摧陷廓清。 

怀中先生言,日本某君以东方思想均不切于实际生活。诚哉其言!吾意即西方思想亦未必尽是,几多之部分,亦应与东方思想同时改造也。今人动教子弟宜立志,又曰某君有志,愚意此最不通。志者,吾有见夫宇宙之真理,照此以定吾人心之所之谓也。今人所谓立志,如有志为军事家,有志为教育家,乃见前辈之行事及近人之施也,羡其成功,盲从以为己志,乃出于一种模仿性。真欲立志,不能如是容易,必先研究哲学、伦理学,以其所得道理,奉以为己身言动之准,立之为前途之鹄,再择其合于此鹄之事,尽力为之,以为达到之方,始谓之有志也。如此之志,方为真志,而非盲从之志。其始所谓立志,只可谓之有求善之倾向,或求真求美之倾向,不过一种之冲动耳,非真正之志也。虽然,此志也容易立哉?十年未得真理,即十年无志;终身未得,即终身无志。此又学之所以贵乎幼也。今人学为文,即好议论,能推断是非,下笔千言,世即誉之为有才,不知此亦妄也。彼其有所议论,皆其心中之臆见,未尝有当于宇宙事理之真。彼既未曾略用研究工夫,真理从何而来?故某公常自谓:“今日之我与昨日之我挑战”,来日之我与今日之我挑战与否,亦未可知。盖研究日进,前之臆见自见其妄也。顾既腾之以为口说,世方以为贤者之言,奉而行矣,今乃知其为妄,宁不误尽天下!弟亦颇有蹈此弊倾向,今后宜戒,只将全副工夫,向大本大源处探讨。探讨既得,自然足以解释一切,而枝叶扶疏,不宜妄论短长,占去日力。阁下以为何如? 

圣人,既得大本者也;贤人,略得大本者也;愚人,不得大本者也。圣人通达天地,明贯过去现在未来,洞悉叁界现象,如孔子之“百世可知”,孟子之“圣人复起,不易吾言”。孔孟对答弟子之问,曾不能难,愚者或震之为神奇,不知并无谬巧,惟在得一大本而已。执此以对付百纷,驾驭动静,举不能逃,而何谬巧哉?(惟宗教家见众人以为神奇,则自神奇之,如耶苏、摩哈默德、释迦牟尼。)欲人人依自己真正主张以行,不盲从他人是非,非普及哲学不可。吾见今之人,为强有力者所利用,滔滔皆是,全失却其主观性灵,颠倒之,播弄之,如商货,如土木,大亦不可哀哉!人人有哲学见解,自然人己平,争端息,真理流行,群妄退匿。 

某君语弟:人何以愚者多而智者少哉?老朽者聪明已蔽,语之以真理而不能听,促之而能动,是亦固然不足怪。惟少年亦多不顾道理之人,只欲冥行,即如上哲学讲堂,只昏昏欲睡,不能入耳。死生亦大矣,此问题都不求解释,只顾目前稊米尘埃之争,则甚矣人之不智!弟谓此种人,大都可悯。彼其不顾道理者,千百年恶社会所陶铸而然,非彼所能自主也,且亦大可怜矣。终日在彼等心中作战者,有数事焉:生死一也,义利一也,毁誉又一也。愚者当前,则只曰于彼乎,于此乎?歧路徘徊,而无一确实之标准,以为判断之主。此如墙上草,风来两边倒,其倒于恶,固偶然之事;倒于善,亦偶然之事。一种笼统之社会制裁,则对于善者鼓吹之,对于恶者裁抑之。一切之人,被驱于此制裁之下,则相率为善不为恶,如今之守节、育婴、修桥、补路,乃至孝、友、睦、雍、任、恤种种之德,无非盲目的动作。此种事实固佳,而要其制裁与被制裁两面之心理,则固尽为盲目的也,不知有宇宙之大本大源也。吾人欲使此愚人而归于智,非普及哲学不可。 

小人累君子,君子当存慈悲之心以救小人。政治、法律、宗教、礼仪制度,及多余之农、工、商业,终日经忙碌,非为君子设也,为小人设也。君子已有高尚之智德,如世但有君子,则政治、法律、礼仪制度,及多余之农、工、商业,皆可废而不用。无知小人太多,世上经营,遂以多数为标准,而牺牲君子一部分以从之,此小人累君子也。然小人者,可悯者也,君子如但顾自己,则可离群索居,古之人有行之者,巢、许是也。若以慈悲为心,则此小人者,吾同胞也,吾宇宙之一体也。吾等独去,则彼将益即于沉沦,自宜为一援手,开其智而蓄其德,与之共跻于圣域。彼时天下皆为圣贤,而无凡愚,可尽毁一切世法,呼太和之气而吸清海之波。孔子知此义,故立太平世为鹄,而不废据乱、升平二世。大同者,吾人之鹄也。立德、立功、立言以尽力于斯世者,吾人存慈悲之心以救小人也。 

弟对于学校甚多不满之处,他日当为书与阁下详论之。现届毕业不远,毕业之后,自思读书为上,教书、办事为下。自揣固未尝立志,对于宇宙,对于人生,对于祖国,对于教育,作何主张,均茫乎未定,如何教书、办事?强而为之,定惟徒费日力,抑且太觉糊涂。以糊涂为因,必得糊涂之果,为此而惧。弟久思组织私塾,采古讲学与今学校二者之长,暂只以叁年为期,课程则以略通国之大要为准。过此即须出洋求学,乃求西学大要,归仍返于私塾生活,以几其深。怀此理想者,4年于兹矣。今距一年之后,即须实行,而基础未立,所忧盖有叁事:一曰人,有师有友,方不孤陋寡闻;二曰地,须交通而避烦嚣;叁曰财,家薄必不能任,既不教书,阙少一分收入,又须费用,增加一分支出,叁者惟此为难。然拟学颜子之箪瓢与范公之画粥,冀可勉强支持也。阁下于此,不知赞否若何?又阁下于自己进修之筹画,愿示规模,作我楷法。 

\sign{润之\quad 1917年8月23日}
\end{xquote}

8月31日,黎锦熙接到了毛泽东8月23日的信,非常高兴。他在日记中写道:

“下午……得润之书,大有见地,非庸碌者。”

1917年9月16日,毛泽东趁着周末休息和张昆弟、彭道良结伴外出野游,他们的目的地是昭山。这昭山乃是潇湘8景之一,有“山市晴岚”之称。

张昆弟在他的日记中生动细致地记述了这次野游活动的前后状况。这种近百年前的史料,尽管后人读起来有点困难,耐不住性子,但因它极其稀少而珍贵,所以笔者就不忍妄加改写,只好抄录如次了。

“今日星期,约与蔡和森、毛润之、彭则厚作一、二日之旅行。早饭后,彭君过河,邀蔡君同至渔湾市会伴,余与毛君先到渔湾市。稍久,彭君一人来,蔡君以值今日移居,不果行。此议发自蔡君,余诺之,并商之于彭、毛二君也。事之难合,诚莫能料。

三人遂沿铁道行,天气炎热,幸风大,温稍解。走十余里,休息于铁路旁边茶店,饮茶解渴,稍坐又行。过十余里,经大托铺。前行六里休息于一饭店,并在此午饭。饭每大碗五十文,菜每碗二十文,三人共吃饭五大碗,小菜五碗。饭后稍息,拟就该店后大塘浴,以水浅不及股,止。遂至店拿行具前行。未及三里,寻一清且深之港坝,三人同浴,余以不善水性甚不自由。浴后,行十四里至目的地,时日将西下矣。遂由山之背缘石砌而上,湘水清临其下,高峰秀挹其上,昭山其名也。

山上有寺,名昭山寺,寺内有和尚三四人。余辈告以来意,时晚,欲在该寺借宿。和尚初有不肯意,余辈遂有作露宿于丛树中之意。和尚后允借宿,露宿暂止。

晚饭后,三人同由山之正面下,就湘江浴。浴后,盘沙对语,凉风缓解,水波助语,不知乐从何来也。久之,由原路上,时行时语,不见山之倒立矣。

和尚待于前门,星光照下,树色苍浓,隐隐生气勃发焉。不久进寺,和尚带余辈至一客房,指旷床为宿处,并借余辈小被一块。

房外有小楼一间,余辈至小楼纳凉,南风乱吹,三人语笑称善者久之。谈话颇久,甚相得也。

毛君云:西人物质文明极盛,遂为衣食住三者所拘,徒供肉欲之发达已耳。若人生仅此衣食住三者而已足,是人生太无价值。

又云:吾辈必想一最容易之方法,以解经济问题,而后求遂吾人理想之世界主义。

又云:人之心力与体力合行一事,事未有难成者。

余甚然其言,且人心能力说,余久信仰,故余有以谭嗣同《仁学》可炼心力之说,友鼎丞然之。

彭君以清夜之惑,久有为僧之志。且云数年后邀余辈同至该邑名山读书,余与毛君亦有此志,毛君之志较余尤坚。余当时亦有感云,风吹树扰声天籁,欲报无从悟弃形。但未出以相示。夜深始睡。”

张昆弟在这篇日记中所说的“彭君以清夜之惑,久有为僧之志”的话,可能是彭道良一时之戏言耳。后来,彭道良不但没有出家当和尚,他还和毛泽东、蔡和森、张昆弟等人一起发起组织了新民学会,并在中国共产党成立之后,加入了中国共产党,为中国革命做出了很多贡献,最终献出了自己的宝贵生命。此乃后话。

且说9月20日,毛泽东和张昆弟等人一起冒着西北风前往水陆洲游泳。

张昆弟在日记中写道:“今日往水陆洲头泅游,人多言西北风过大,天气太冷。余等全行不顾,下水亦不觉冷,上岸也不见病。坚固皮肤,增进血液,扩充肺腑,增加力气,不得不谓运动中最有益者。人言固不足信哉!”

9月23日,张昆弟日记中记述了他和毛泽东、蔡和森在22日下午及23日的活动:

“昨日下午与毛君润之游泳。游泳后至麓山蔡和森君居。时将黄昏,遂宿于此。夜谈颇久。

毛润之云:现在国民性惰,虚伪相崇,奴隶成性,思想狭隘。安得国人有大哲学革命家,大伦理革命家,如俄之托尔斯泰其人,以洗涤国民之旧思想,开发其新光明思想。余甚然其意。中国人沉郁固塞,陋不自知,入主出奴,普成习性。安得有俄之托尔斯泰其人者,冲决一切现象之网罗,发展其理想之世界,行之以身,著之以书,以真理为归,真理所在,毫不旁顾。前之谭嗣同,今之陈独秀,其人者魄力颇雄大,诚非今日俗学所可比拟。

又,毛君主张将唐宋以后之文集、诗集,焚诸一炉,又主张家族革命,师生革命;革命非兵戎相见之谓。乃除旧布新之谓。

今日早起,同蔡、毛二君由蔡君居侧上岳麓,沿山脊而行,至书院后下山。凉风大发,空气清爽。空气浴、大风浴,胸襟洞彻,旷然有远俗之慨。归时十一点钟矣。” 

9月26日,毛泽东同班同学罗学瓒在他的日记中记载:

“余借毛君泽东手录西洋伦理学七本,自旧历六月底阅起,于今日阅毕。”

9月30日这一天是中国传统的中秋节。

是日晚,毛泽东约同张昆弟、贺培真、罗学瓒、陈绍休等同学,来到长沙水陆洲的船上赏月,有同学提议比赛背诵唐诗中带有“月”字的句子,说是看谁背得最多,赢者当为船长,输者就罚他划桨。毛泽东首先吟诵道:

“青天有月来几时”、“欲上青天揽明月”、“今人不见古时月”、“今月曾经照古人”、“昨夜月同行”、“游人日月长”、“中天悬明月”、“月是故乡明”……

众人数着句子,毛泽东竟然一口气背诵出100多句。同学们无不为他超凡的记忆力所折服,都忍不住鼓掌叫起好来。

从1917下学期开始,一师本科毕业班开设了修身课,由杨昌济先生讲授。他用的教材是蔡元培翻译的德国康德派哲学家、伦理学家泡尔生所著的《伦理学原理》。

毛泽东很喜欢先生的修身课,他不但认真听课,而且还利用课余时间细读了教材。在这本12万字的《伦理学原理》中,他逐字逐句研读,分别用红笔、墨笔画下了很多圈点、单杠、双杠、三角、叉子等等符号,还在空白处写了150余条多达12000多字的批语,其中有的批语长约800多字。这些批语的内容,绝大部分是针对书中的一些哲学、伦理学观点,提出了他自己的看法。他强调个人价值,主张唯我论,提倡个性解放,反对封建传统观念,反对专制主义的束缚和压抑。由于受这本教科书的启迪,他又写出了一篇题目叫作《心之力》的作文。

据斯诺在《西行漫记》中记载,毛泽东是这样说的:

“给我印象最深的教员是杨昌济,他是从英国回来的留学生,后来我同他的生活有密切的关系。他教授伦理学,是一个唯心主义者,一个道德高尚的人。他对自己的伦理学有强烈信仰,努力鼓励学生立志做有益于社会的正大光明的人。我在他的影响之下,读了蔡元培翻译的一本伦理学的书。我受到这本书的启发,写了一篇题为《心之力》的文章。那时我是一个唯心主义者,杨昌济老师从他的唯心主义观点出发,高度赞赏我的那篇文章。他给了我一百分。”

时光流转,日月如梭,到了21世纪初,有好事者朗誉林先生在互联网上发出了早已佚失的《心之力》的几个版本。后来网友黄铭在2015年10月13日则发表了一篇《关于托名毛泽东的网文<心之力>的全面证伪》的长文(可在百度上查阅原文),很有说服力。这些资料,笔者都看到了,但感觉到朗誉林提供的《心之力》毕竟还有针砭时弊、警醒世人的功效,便将几个版本收集起来,进行了认真的校对、编辑,在本传中引用了,并在章末发了几句议论。如今经网友张哲提醒,这样的托名之作放在毛泽东名下,毕竟大不敬,便全部删去了。

张哲同志在微信中告诉我,网传《心之力》一文,与毛泽东同一时期写给黎锦熙老师的长信相比,差异极大,由此可证其伪。于是,我便将《心之力》与毛泽东写于1917年8月23日(见本传上文)的长信,进行了认真比对,这才发现,无论是在文字、文风方面,还是在思想性方面,两者都是不可同日而语的。当然,《心之力》一文中的一些内容,可能出自毛泽东的读书笔记和读《伦理学原理》的批语,这就需要后人考证鉴别了。

再说此时的毛泽东特别喜欢社会科学,而对那些自然科学门类并不十分感兴趣,所以他反对学校把自然科学列为必修课。因此他将功夫尽数花在了社会科学上,专心于哲学、史地、文学等方面的研究。他博览群书,凡是能够搜集到的古今中外的各种名著,如诸子百家、诗词歌赋、稗官小说、近人文集等等,无不浏览。他经济拮据,没有多少钱可以去买自己喜欢的书,就只能用省吃俭用的节余去买一些折价的书,而他大量阅读的那些书籍基本上都是向老师和同学们借来的,他非常珍视这些资料,常常把他喜爱的内容抄录保存下来。

坚持不懈,持之以恒,是毛泽东读书学习的突出特点。他每天从早到晚,读书不止。

毛泽东还非常注重自学,课堂上有限的时间满足不了他的求知欲望,于是便精心安排了一个自学计划。晚上熄灯后,别人都休息了,他却独自坐在走廊的路灯下或者茶炉房里,借着微弱的灯光,捧着书本苦读到深夜,有时甚至是通宵不眠。

毛泽东还有一个特立独行之处,“为静中求学”,他只身到学校后面的妙峰山顶读书,为“闹中求静”,他则到车水马龙的长沙城南门口读书,藉此磨砺意志,锻炼在任何环境中都能够专心致志地学习和思考问题的本领。

功夫不负有心人,毛泽东的这个习惯,后来在战争年代果然派上了大用场,冷静思考,处乱不惊,任是炮火连天,他也能泰然若素。

1917年10月间,第一师范的学生团体学友会进行了改选。毛泽东被选为总务、兼教育研究部部长。周世钊也同时被选为文学部部长。在此之前,一师学友会的总务和各部部长均由学监和教员们充任,这次改选开创了由学生担任总务和部长的先例。

毛泽东上任后做了大量工作,以极大的热情使学友会愈发活跃起来,显示他那出色的组织才能。

10月13日,学友会召开会议,到会的有62人,主要议程是讨论各项会务。毛泽东在会议上提出了本届学友会工作的八项议案,被与会者通过。这八项议案是:

一、本届会金如何征集;二、开展演讲日期;三、各部开展演讲的次数;四、教员聘定;五、预算编制;六、成绩保存;七、作学友会纪事录;八、筹设学友会图书馆。

10月14日,学友会召开各部部长会议,毛泽东又提出六项议案,内容为:

确定部员,聘定教员,添派录事,规定考题,开演次数,进行程序。

毛泽东积极倡导和组织各种课余的学术研究活动和体育活动,督促各部制订切实可行的学术研讨计划和体育活动计划,为各部聘请义务指导教师,认真督导各部举办同学演讲辩论和各种专门的学术报告会,并邀请名人演讲。

从1917年10月15日至11月16日,在32天内,学友会各部的活动就多达64项。

毛泽东注重图书资料工作。他精打细算,从学友会经费中挤出一些钱,购买一些新杂志,如《新青年》、《太平洋》、《科学》、《旅欧杂志》、《教育周报》、《教育研究》等,把学友会图书馆办得相当不错。

1917年间,湖南一师虽然实行了10分钟课间操制度,但由于组织得不是很好,同学们的锻炼只是徒有虚名。在一场可怕的传染病蔓延之时,8班傅传甲等7位同学染病后不治身亡。毛泽东在学校举行的追悼会上,为亡友写了一副挽联:

为何死了七位同学?只因不习十分间操。

此后,他利用学友会总务的职务,大力倡导体育运动,并主持开展了全校性的各种体育活动。针对学校雨天没有运动场地、晚上不能开展体育活动这一状况,他积极创造条件,发动大家打乒乓球健身。

在毛泽东主持下,学友会制做了一些乒乓球架,做了12片竹布网,分发各班,并指定学友会事务室、礼堂、会客室、洋楼等处为打球地点。自此之后,“乒乓之声一时聒耳,或谓之乒乓狂云。”

一师各种课外活动的广泛开展,促进了同学们在德育、智育、体育3个方面的全面发展。

毛泽东在主持一师学友会工作期间,还特别推崇徐特立先生编写的《教育学》和《各科教学法》中的《平民教育》、《如何举办夜校》等章节,满腔热忱地主办了工人夜学。

此时的一师附近,集中了长沙几个新式企业,有电灯公司、造币厂、黑铅炼石厂、铜元局等。此外,粤汉铁路武昌至长沙段、长沙至株洲段的工人,也都住在这一带。那些产业工人、人力车夫、小商小贩及其他劳动者,生活都十分贫困,绝大多数人没有受教育的机会,基本上都是文盲或半文盲。第一师范的教职员们为了给穷人扫盲,在1917年上半年开办了一所夜学,但是办得并不是很成功。

学友会改造以后,毛泽东总结了办夜学的经验教训,认为工人夜学还应当继续办,而且还要办好。在学友会研究工作计划时,毛泽东说:

“师范本以教育为天职。我国现状,社会之中坚实为大多数失学之国民,此辈阻碍政令之推行、自治之组织、风俗之改良、教育之普及,其力甚大。此时固应以学校教育为急,造成新国民及有开拓能力之人材。而欲达此目的,不可不去为此目的之阻碍。”

“欧美号称教育普及,而夜学与露天学校、半日学校、林间学校等不废;褓姆有院,聋盲有院,残废有院,精神病者有院,于无可教育之中,求其一线之可教者,而不忍恝置也。”

帮助偶因天赋之不济,境遇之不同而失学者,“正仁人之所宜矜惜,而无可自诿者。”

“设此夜学可为吾等实习之场,与工业之设工场,商校之设商市,农校之设农场相等。”

“现时学校大弊,在与社会打成两橛,犹鸿沟之分东西。一入学校,俯视社会犹如登天;社会之于学校,亦视为一种神圣不可捉摸之物。”而兴办夜学,正可以“疏通隔阂,社会与学校团结一气。”长此以往,“社会之人皆学校毕业之人,学校之局部为一时之小学校,社会之全体实为永久之大学校。”

毛泽东还提出,工人夜学要由一师三四年级的学生来承办,归学友会教育研究部负责。

他的意见得到了学友会和教职员们的一致赞成,并推举由周世钊担任夜校的管理员。

学友会立即展开了紧张的工人夜校筹备工作,招生广告编印出来以后,“除张贴通衢外,并函托警察分发,令国民学校学生带归劝告,久之无效”。 

第二次的工人夜学招生广告编印出来后,“除印刷分贴外,另书大张张贴显处,亦无结果,报名者并前次9人而已。”

毛泽东马上召集校友会部分成员开会,分析这两次招生均未成功的原因。他认为:

一、请工人来夜学读书,不要钱,这是一桩新鲜事,“盖社会不熟悉学校内容,虽有广告,疑不敢即入”。

二、“仅仅张贴,无人注意,彼等不注意于此,犹之吾人不注意官府布告也。”“并未遍散,彼等未能手受而目击之。”

三、让警察分发也不妥,“其已否奉行属疑问;即分发矣,人民视警察俨然官吏,久已望而畏之,更何能信?”

10月30日,毛泽东亲自用白话文起草了第3份《夜学招学广告》,他是这样写的:

列位工人来听我们说几句白话:

列位最不便利的是甚么,大家晓得吗?就是俗话说的,讲了写不得,写了认不得,有数算不得。都是个人,照这样看来,岂不是同木石一样!所以大家要求点知识,写得几个字,认得几个字,算得几笔数,方才是便利的。列位做工的人,又要劳动又无人教授,如何才能写得几个字,算得几笔数呢?现今有个最好的法子,就是我们第一师范办了一个夜学。今年上半年学生很多,列位中想有听到过的。这个夜学专为列位工人设的,从礼拜一起至礼拜五止,每夜上课两点钟,教的是写信、算账,都是列位时刻要用的。讲义归我们发给,并不要钱。夜间上课又于列位工作并无妨碍。若是要来求学的,就赶快于一礼拜内到师范的号房报名。

列位大家想想,我们为什么要如此做?无非是念列位工作的苦楚,想列位个个写得、算得。列位何不早来报个名,大家来听听讲?有说时势不好,恐怕犯了戒严的命令,此事我们可以担保,上学以后,每人发听讲牌一块,遇有军警查问,说是师范夜学学生就无妨了。若有为难之处,我们替你做保,此层只管放心的。快快来报名,莫再耽搁!

这份白话广告印好后,毛泽东组织一批同学拿着数百份一边到处张贴,一边挨家挨户分发,进行口头宣传。从“铜元局一带,铁路两旁到洪恩寺一带,左至大椿桥,右至社坛岭、天鹅塘,共发去600张,”受到工人们的热烈欢迎,他们争相询问夜学的情况,奔走相告,兴高采烈地说:“读夜书去!”不到3天时间,报名的工人及其子弟就有102名。

这正是:\begin{xemph}身处草莽论天下,闹市读书有几人?

\hspace{4em}白话巧请夜学者,满座皆是褴褛身。\end{xemph}


    欲知毛泽东如何办好工人夜校,请看下一章。

    东方翁曰:毛泽东点评近现代历史人物,认为段祺瑞、黎元洪、张勋、冯国璋等官僚政客,你争我夺,人人欲得天下,造成了“天下纷纷”的局势,这些乱世奸雄皆不知“倡学”,故“本源未得”。而曾国藩镇压太平天国最厉害的一手,就是以倡学攻心。一场地主阶级与农民阶级之间的战争,却被曾国藩巧妙地说成是一场维护名教、保卫中国传统文化的战争,真可谓是匠心独运。所以,毛泽东曾在一封信中说:“愚于近人,独服曾文正,观其收拾洪杨一役,完满无缺,使以今人易其位,其能如彼之完满乎?”斯言诚为智者之识也。
\end{document}
